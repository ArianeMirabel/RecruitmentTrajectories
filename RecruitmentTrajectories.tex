%%%%%%%%%%%%%%%%%%%%%%%%%%%%%%%%%%%%%%%%%
% Article EcoFoG
% Version 2.1 (23/10/2017)
%
% adapté de :
% Stylish Article
% LaTeX Template
% Version 1.0 (31/1/13)
%
% This template has been downloaded from:
% http://www.LaTeXTemplates.com
%
% Original author:
% Mathias Legrand (legrand.mathias@gmail.com)
%
% License:
% CC BY-NC-SA 3.0 (http://creativecommons.org/licenses/by-nc-sa/3.0/)
%
%%%%%%%%%%%%%%%%%%%%%%%%%%%%%%%%%%%%%%%%%


%----------------------------------------------------------------------------------------
%	PACKAGES AND OTHER DOCUMENT CONFIGURATIONS
%----------------------------------------------------------------------------------------

\documentclass[fleqn,10pt]{ArtEcoFoG} % Document font size and equations flushed left

\setcounter{tocdepth}{3} % Show only three levels in the table of contents section: sections, subsections and subsubsections


% Pandoc environments
\usepackage{framed}
\usepackage{fancyvrb}
\providecommand{\tightlist}{%
  \setlength{\itemsep}{0pt}\setlength{\parskip}{0pt}}
\newcommand{\VerbBar}{|}
\newcommand{\VERB}{\Verb[commandchars=\\\{\}]}
\DefineVerbatimEnvironment{Highlighting}{Verbatim}{commandchars=\\\{\}, fontsize=\scriptsize} % Code R
\definecolor{shadecolor}{RGB}{248,248,248}
\newenvironment{Shaded}{\begin{snugshade}}{\end{snugshade}}
\newcommand{\KeywordTok}[1]{\textcolor[rgb]{0.13,0.29,0.53}{\textbf{{#1}}}}
\newcommand{\DataTypeTok}[1]{\textcolor[rgb]{0.13,0.29,0.53}{{#1}}}
\newcommand{\DecValTok}[1]{\textcolor[rgb]{0.00,0.00,0.81}{{#1}}}
\newcommand{\BaseNTok}[1]{\textcolor[rgb]{0.00,0.00,0.81}{{#1}}}
\newcommand{\FloatTok}[1]{\textcolor[rgb]{0.00,0.00,0.81}{{#1}}}
\newcommand{\ConstantTok}[1]{\textcolor[rgb]{0.00,0.00,0.00}{{#1}}}
\newcommand{\CharTok}[1]{\textcolor[rgb]{0.31,0.60,0.02}{{#1}}}
\newcommand{\SpecialCharTok}[1]{\textcolor[rgb]{0.00,0.00,0.00}{{#1}}}
\newcommand{\StringTok}[1]{\textcolor[rgb]{0.31,0.60,0.02}{{#1}}}
\newcommand{\VerbatimStringTok}[1]{\textcolor[rgb]{0.31,0.60,0.02}{{#1}}}
\newcommand{\SpecialStringTok}[1]{\textcolor[rgb]{0.31,0.60,0.02}{{#1}}}
\newcommand{\ImportTok}[1]{{#1}}
\newcommand{\CommentTok}[1]{\textcolor[rgb]{0.56,0.35,0.01}{\textit{{#1}}}}
\newcommand{\DocumentationTok}[1]{\textcolor[rgb]{0.56,0.35,0.01}{\textbf{\textit{{#1}}}}}
\newcommand{\AnnotationTok}[1]{\textcolor[rgb]{0.56,0.35,0.01}{\textbf{\textit{{#1}}}}}
\newcommand{\CommentVarTok}[1]{\textcolor[rgb]{0.56,0.35,0.01}{\textbf{\textit{{#1}}}}}
\newcommand{\OtherTok}[1]{\textcolor[rgb]{0.56,0.35,0.01}{{#1}}}
\newcommand{\FunctionTok}[1]{\textcolor[rgb]{0.00,0.00,0.00}{{#1}}}
\newcommand{\VariableTok}[1]{\textcolor[rgb]{0.00,0.00,0.00}{{#1}}}
\newcommand{\ControlFlowTok}[1]{\textcolor[rgb]{0.13,0.29,0.53}{\textbf{{#1}}}}
\newcommand{\OperatorTok}[1]{\textcolor[rgb]{0.81,0.36,0.00}{\textbf{{#1}}}}
\newcommand{\BuiltInTok}[1]{{#1}}
\newcommand{\ExtensionTok}[1]{{#1}}
\newcommand{\PreprocessorTok}[1]{\textcolor[rgb]{0.56,0.35,0.01}{\textit{{#1}}}}
\newcommand{\AttributeTok}[1]{\textcolor[rgb]{0.77,0.63,0.00}{{#1}}}
\newcommand{\RegionMarkerTok}[1]{{#1}}
\newcommand{\InformationTok}[1]{\textcolor[rgb]{0.56,0.35,0.01}{\textbf{\textit{{#1}}}}}
\newcommand{\WarningTok}[1]{\textcolor[rgb]{0.56,0.35,0.01}{\textbf{\textit{{#1}}}}}
\newcommand{\AlertTok}[1]{\textcolor[rgb]{0.94,0.16,0.16}{{#1}}}
\newcommand{\ErrorTok}[1]{\textcolor[rgb]{0.64,0.00,0.00}{\textbf{{#1}}}}
\newcommand{\NormalTok}[1]{{#1}}
\usepackage{longtable,booktabs}
\usepackage{caption}
% These lines are needed to make table captions work with longtable:
\makeatletter
\def\fnum@table{\tablename~\thetable}
\makeatother
% longtable 2 columns
% https://tex.stackexchange.com/questions/161431/how-to-solve-longtable-is-not-in-1-column-mode-error
\makeatletter
\let\oldlt\longtable
\let\endoldlt\endlongtable
\def\longtable{\@ifnextchar[\longtable@i \longtable@ii}
\def\longtable@i[#1]{\begin{figure}[t]
\onecolumn
\begin{minipage}{0.5\textwidth}\scriptsize
\oldlt[#1]
}
\def\longtable@ii{\begin{figure}[t]
\onecolumn
\begin{minipage}{0.5\textwidth}\scriptsize
\oldlt
}
\def\endlongtable{\endoldlt
\end{minipage}
\twocolumn
\end{figure}}
\makeatother

\usepackage{graphicx,grffile}
\makeatletter
\def\maxwidth{\ifdim\Gin@nat@width>\linewidth\linewidth\else\Gin@nat@width\fi}
\def\maxheight{\ifdim\Gin@nat@height>\textheight0.8\textheight\else\Gin@nat@height\fi}
\makeatother
% Scale images if necessary, so that they will not overflow the page
% margins by default, and it is still possible to overwrite the defaults
% using explicit options in \includegraphics[width, height, ...]{}
\setkeys{Gin}{width=\maxwidth,height=\maxheight,keepaspectratio}

% User-adder preamble
\usepackage{textcomp} \DeclareUnicodeCharacter{B0}{\textdegree}
\usepackage{tabu}
\renewenvironment{table}{\begin{table*}}{\end{table*}\ignorespacesafterend}
\hyphenation{bio-di-ver-si-ty sap-lings}

%----------------------------------------------------------------------------------------
%	ARTICLE INFORMATION
%----------------------------------------------------------------------------------------

\JournalInfo{Hal 00679993} % Journal information
\Archive{DOI xxxx} % Additional notes (e.g. copyright, DOI, review/research article)

\PaperTitle{30 Years of Recruitment in Tropical Forest After Selective Logging} % Article title

\Authors{
Ariane MIRABEL\textsuperscript{1*}\\ Eric MARCON\textsuperscript{1}\\ Bruno HERAULT\textsuperscript{2}
} % Authors
\affiliation{
\textsuperscript{1}UMR EcoFoG, AgroParistech, CNRS, Cirad, INRA, Université des Antilles,
Université de Guyane.\\ \hspace{1em} Campus Agronomique, 97310 Kourou, France.\\\textsuperscript{2}INPHB (Institut National Ploytechnique Félix Houphoüet Boigny)\\ \hspace{1em} Yamoussoukro, Ivory Coast
}
\affiliation{*\textbf{Contact}: ariane.mirabel@ecofog.gf, http://www.ecofog.gf/spip.php?article47} % Corresponding author

\Keywords{mot-clés, séparés par des virgules} % Keywords - if you don't want any simply remove all the text between the curly brackets
\newcommand{\keywordname}{Mots-clés} % Defines the keywords heading name

%----------------------------------------------------------------------------------------
%	ABSTRACT
%----------------------------------------------------------------------------------------

\Abstract{
Résumé de l'article.
}

%----------------------------------------------------------------------------------------

\begin{document}

\selectlanguage{english}

\flushbottom % Makes all text pages the same height

\maketitle % Print the title and abstract box

\tableofcontents % Print the contents section

\thispagestyle{empty} % Removes page numbering from the first page

%----------------------------------------------------------------------------------------
%	ARTICLE CONTENTS
%----------------------------------------------------------------------------------------


\section{Introduction}\label{introduction}

Determining the response of tropical forests to disturbance is a key to
predict their fate in a global change context. In that respect a vast
literature has successfully modeled the response of tropical forest
dynamics, carbon stocks and fluxes to anthropogenic and natural
disturbances
\citep{Gourlet-Fleury2000, Putz2012, Martin2015, Piponiot2016}.
Regarding diversity, however, similar attempts have been hindered by
both the huge biological diversity and the scarcity of long-term
monitoring. If the response to disturbance has been identified for
common species assemblages, it usually remained confined to few
commercial and valuable species
\citep{Sebbenn2008, Rozendaal2010, Vinson2015}. Forest dynamics, though,
result from the constantly evolving interactions and feedbacks among
trees and their environment and requires to encompass complete
communities \citep{DeAvila2016}. Forests response to disturbance is
build on the recruitment of new individuals and on the mortality of
disturbance survivors. Disturbance survivors proved to mirror the
pre-disturbance forest composition \citep{Piponiot2018}, so now
predicting the recruitment composition and diversity trajectories would
be a major step to elucidate the future of tropical forest in a changing
global environment where disturbance are expected to become more and
more frequent. This would give insights into the resilience of this
hyperdiverse ecosystems, elucidate the determinism, or not, of tropical
forests trajectories, test the convergence after disturbance of
taxonomic and functional communities towards initial state and also help
future adaptative conservation strategies
\citep{Diaz2005, Gardner2007, Schwartz2017}.

Recruitment processes result from the interplay of deterministic
processes, like the exclusion of stress-intolerant species or the
limitation of similarity through the competition for resource
\citep{Ackerly2003, McGill2006}, and stochastic processes like random
dispersal, recruitment and death \citep{Hubbell2001}. Supposedly,
deterministic processes are inherently linked to disturbance which
locally changes ecosystem's biotic and abiotic conditions and maintains
large ecological range of species in the community
\citep{Molino2001, Bongers2009}. These processes would, inter alia,
maintain in communities fast growing species with efficient resource
acquisition but poorly resistant to hazards and diseases while in mature
forests favor slow-growing, stress tolerant species conserving resources
in long-lived dense tissues \citep{Denslow1980}. Deterministic processes
therefore rely on the Intermediate Disturbance Hypothesis (IDH) that
explains the maintenance of tropical forests biodiversity by the patchy
variability of environmental conditions in space and time
\citep{Guitet2018}. Specifically, in tropical wet forests changes in
light availability after disturbance enhance the recruitment of pioneers
and light-demanding species in comparison with mature stands forests
dominated by more competitive shade bearers. Empirical tests of the IDH
in tropical rainforests, though, proved hard to succeed and yielded
controversial results \citep{Hubbell1999, Molino2001, Sheil2003}. During
post-disturbance times, the shift from resource-acquisitive to
resource-conservative ecological strategies may be detected in leaves
(leaf thickness, toughness, chlorophyll content and specific area) and
stem (wood specific gravity and bark thickness) functional traits and in
life-history traits (maximum height at adult stage and class of seed
mass) \citep{Wright2004, Chave2009b, Herault2011}.

In this paper we follow the fate of a recruited tree communities (60121
individuals) over 30 years on a disturbance gradient, with 10 to 60\% of
forest biomass removed. We assessed the taxonomic as well as functional
diversity of recruited trees, using a large functional trait database
covering the leaf, wood and life-history spectra. We compared the
observed trajectories to null models representing random trees
recruitment and randomized functional traits. We aimed to (i) assess the
role of deterministic processes compared to stochastic recruitment after
disturbance, (ii) assess the taxonomic and functional convergence of
forest communities and the maintenance of taxonomic composition in the
long term, and (iii) determine the resilience of the ecosystem.

\section{Material and Methods}\label{material-and-methods}

\subsection{Study Site}\label{study-site}

The Paracou station is located in a lowland tropical rain forest in
French Guiana (5°18'N and 52°53'W). Climate is tropical wet with mean
annual precipitation averaging 2980 mm.y\textsuperscript{-1} (30-y
period) and a 3-months dry season (\textless{} 100
mm.months\textsuperscript{-1}) from mid-August to mid-November, and a
one-month dry season in March \citep{Wagner2011}. Elevation ranges from
5 to 50 m and mean annual temperature is 26°C. Soils are thin acrisols
over a layer of transformed saprolite with low permeability generating
lateral drainage during heavy rains. The disturbance experiment spread
over a network of twelve 6.25ha plots (Table \ref{tab:Tab1}) that
underwent three disturbance treatments in 1986-1987 \citep{Herault2018}.
Dominant families are Fabaceae, Chrysobalanaceae, Lecythidaceae and
Sapotaceae.

\begin{table}

\caption{\label{tab:Tab1}Intervention table, summary of the disturbance intensity for the 4 plot treatments in Paracou.}
\centering
\begin{tabu} to \linewidth {>{\raggedright}X>{\raggedright}X>{\raggedright}X>{\raggedright}X>{\raggedright}X}
\toprule
Treatment & Timber & Thinning & Fuelwood & \%AGB lost\\
\midrule
Control &  &  &  & 0\\
T1 & DBH $\geq$ 50 cm, commercial species, $\approx$ 10 trees/ha &  &  & $[12\%-33\%]$\\
T2 & DBH $\geq$ 50 cm, commercial species, $\approx$ 10 trees/ha & DBH $\geq$ 40 cm, non-valuable species, $\approx$ 30 trees/ha &  & $[33\%-56\%]$\\
T3 & DBH $\geq$ 50 cm, commercial species, $\approx$ 10 trees/ha & DBH $\geq$ 50 cm, non-valuable species, $\approx$ 15 trees/ha & 40 cm $\leq$ DBH $\leq$ 50 cm, non-valuable species, $\approx$ 15 trees/ha & $[35\%-56\%]$\\
\bottomrule
\end{tabu}
\end{table}

\subsection{Inventories Protocol and Dataset
Collection}\label{inventories-protocol-and-dataset-collection}

All trees above 10 cm DBH were mapped and measured annually since 1984.
During inventories, trees were first identified with a vernacular name
assigned by the field team, and afterward with a scientific name
assigned by a botanist during regular botanical campaigns. Botanical
campaigns have been carried out every 5 to 6 years from 2003 onwards.
This raised methodological issues as vernacular names usually correspond
to different botanical species, resulting in significant taxonomic
uncertainties that were propagated to composition and diversity metrics.
Vernacular names were replaced through multinomial trials
\(M_v\Big(\big[s_1, s_2, …, s_N\big],\big[\alpha_1, \alpha_2,…, \alpha_3\big]\Big)\)
based on the observed association probability
\(\big[\alpha_1, \alpha_2,…, \alpha_3\big]\) between each vernacular
name \emph{v} and the species \(\big[s_1, s_2, …, s_N\big]\) recorded in
the inventory. See appendix 1 and \citet{Aubry-Kientz2013} for the
detailed methodology. To avoid remaining identification caveats, the
simulated botanical inventories were reported at genus level.

Six functional traits, representing leaf economics (leaves thickness,
toughness, total chlorophyll content and specific leaf area, the leaf
area per unit dry mass), wood economics (wood specific gravity and bark
thickness) and life history traits (maximum specific height and seed
mass), come from the BRIDGE project \footnote{http://www.ecofog.gf/Bridge/}
where trait values were measured on nine french guianan forest plots,
including two in Paracou. Missing trait values (10\%) were filled using
multivariate imputation by chained equation (mice). As traits
variability was lower within species and within genus, we accounted for
the phylogenetic signal of the functional traits in restricting thegap
filling processes to samples pertaining to the next higher taxonomic
level (refs MICE). As seed mass information corresponds to a
classification into mass classes, no data filling process was applied so
analysis were performed only considering the 414 botanical species of
the seed mass dataset.

Functional trajectories were estimated with the Rao quadratic entropy
using community weighted means (CWM) \citep{Diaz2007, Garnier2004}. Seed
mass trajectories were reported by the proportion of each class recorded
in the inventories. All composition and diversity metrics are the
average obtained after 50 iterations of taxonomy and trait values
uncertainty propagation.

\subsection{Recruitment trajectories}\label{recruitment-trajectories}

We split the forest community in `survivors, i.e.~trees that survived
the disturbance, and post-disturbance recruited trees. Two recruitment
metrics were examined: on the one hand the ``punctual recruitment'' by
2-year intervals after disturbance, on the other hand all recruited
trees since disturbance, hereafter ``accumulated recruits''. The
taxonomic diversity was assessed through Richness and the Hill number
translation of Shannon and Simpson indices
\citep{Hill1973, chao2015estimating, Marcon2015b}. These three indices
belong to the set of HCDT or generalized entropy, respectively
corresponding to the 0, 1 and 2 order of diversity (\emph{q}), which
grasps the balance between richness and evenness in the community, with
common species weighting more than rare ones when \emph{q} increases.
The similarity between the recruited trees and the pre-disturbance
forest was measured with the turnover metrics detailed in
\citet{Podani2013a}. To determine whether recruitment trajectories
ensued from a pure random process, observed trajectories were compared
to those generated by 50 repetitions of a random null model shuffling
individuals among plots while preserving species abundance and plots'
tree density.

To draw plots trajectories we applied a moving average with a one step
window allowing to mitigate the heterogeneity of inventory protocols
between years.

\section{Results}\label{results}

\subsection{Recruitment Diversity}\label{recruitment-diversity}

\subsubsection{Taxonomic Diversity}\label{taxonomic-diversity}

Punctual recruits' diversity followed a consistent trajectory among
disturbance treatments with first higher richness and lower evenness
than in control plots and then equivalent richness and lower evenness
(Figure (\ref{fig:Fig1}). For recruits accumulated since disturbance,
the richness (order 0) in highly disturbed plots (T3 and some T2) was
higher than in control plots, consistently with the increase of
recruited trees after disturbance, and the evenness (order 2) was lower,
specifically for the most disturbed plots (Appendix I, fig. S1).

\begin{figure*}

{\centering \includegraphics[width=0.8\linewidth]{RecruitmentTrajectories_files/figure-latex/Fig1-1} 

}

\caption{Trajectories of Richness, Shannon and Simpson diversity for 2-years laps punctual  recruitment (upper panels) and divergence to null model (lower panels). Lines colors refer to the perturbation regime: green for control, blue for T1, orange for T2 and red for T3 disturbance treatments. Plain lines correspond to the median observed after uncertainty propagation and are given along with the 95\% confidence interval (grey envelope).}\label{fig:Fig1}
\end{figure*}

Punctual and accumulated recruitment diversity of orders 0, 1 and 2 were
then compared to a null random recruitment model. In control plots the
richness (order 0) and evenness (order 2) of punctual recruits remained
equivalent or higher than for the null random model. For all disturbed
plots in contrast both richness and evenness were lower than these of a
random null model but displayed a significant but unachieved
humped-shaped trajectory for all plots (Figure \ref{fig:Fig1}).
Accumulated recruitment richness and evenness were higher or equivalent
to those of the null model for plots T1 and some plots T2 but lower for
plots T3 and a plot T2 (AppendixI, fig. S1).

\subsubsection{Functional Diversity and
Composition}\label{functional-diversity-and-composition}

The functional diversity (Rao diversity) of punctual recruitment was
measured and compared to a null model of random traits shuffling. In
most distrubed plots (plots T2 and T3) the functional diversity was
deacrinsing and lower to this of control plots until 15 years after
disturbance (Figure \ref{fig:Fig3}). It then increased to values
equivalent or higher to those observed in control plots. For all
disturbed and control plots the observed functional diversity was lower
than for the null model of random traits shuffle, except for two T1
plots.

\begin{figure}

{\centering \includegraphics{RecruitmentTrajectories_files/figure-latex/Fig2-1} 

}

\caption{Functional diversity of punctual recruited trees from the considered functional traits and divergence to null model. Values reported correspond to the plot-level median and the 95\% confidence interval obtained after 50 repetition of the taxonomic uncertainty propagation and the functional database gap-filling processes and 50 run of the null model. Lines colors correspond to the logging treatment initially applied (green for control, blue for T1,orange for T2 and red for T3).}\label{fig:Fig2}
\end{figure}

Trajectories of recruited trees in the functional spaces showed the
dominance after disturbance of species displaying large exchange surface
area and light tissues (high SLA, low leaf toughnessand thickness and
low wood specific gravity) (Figure \ref{fig:Fig3}). All traits
trajectories displayed univariate CWM trajectories with leaf toughness,
wood specific gravity and bark thickness decreasing before stabilizing
at low values around 15 after disturbance, except SLA and leaf thickness
that displayed a unimodal trajectory with a maximum reach around 15
years after disturbance.

\begin{figure*}

{\centering \includegraphics[width=0.8\linewidth]{RecruitmentTrajectories_files/figure-latex/Fig3-1} 

}

\caption{Community weighted means (CWM) of the four disturbance treatment for the four leaf traits, the two stem traits  and the specific Hmax. Values reported correspond to the plot-level median obtained after 50 repetition of the taxonomic uncertainty propagation and the functional database gap-filling processes. Lines colors correspond to the disturbance intensity (green for control, blue for T1,orange for T2 and red for T3).}\label{fig:Fig3}
\end{figure*}

\subsection{Recruitment Turnover}\label{recruitment-turnover}

In control plots species turnover remained highly stable for the 30
sampled years (Figure \ref{fig:Fig4}), reflecting a strong similarity
between the initial plots composition and the punctual recruits. In
disturbed plots, turnover displayed a unimodal response to disturbance,
with maximum reached around 15 years and with a value positively
correlated to the disturbance intensity (\(\rho_{spearman}=0.93\)). The
turnover trajectory returned close to zero for all plots 30 years after
disturbance.

\begin{figure}

{\centering \includegraphics{RecruitmentTrajectories_files/figure-latex/Fig4-1} 

}

\caption{Trajectories over the 30 sampled years of the abundance-based turnover between recruited trees and intial communities before disturbance. Grey envelopes correspond to the 0.025 and 0.975 percentiles of the uncertainty propagation procedue and lines to the median in green for control, blue for T1,orange for T2 and red for T3).}\label{fig:Fig4}
\end{figure}

\section{Discussion}\label{discussion}

\subsection{On the underlyings of the hump-shaped
trajectories}\label{on-the-underlyings-of-the-hump-shaped-trajectories}

The trajectories of punctual recruitment richness, some key functional
traits (SLA and bark thickness) and the species turn-over exhibited
hump-shaped, unimodal trajectories.

The 10-15 first years of these trajectories seemed driven by the growth
of pre-disturbance saplings benefiting from the environmental changes
and alleviated competition that follow disturbance \citep{Herault2010}.
After low disturbance intensity this translated into a stable functional
diversity of the recruited community, equivalent to these of control
plots governed by stochastic recruitment processes. After intense
disturbance, this phase corresponded to sharp increase of SLA and
decrease of wood density and leaf thickness community means. This
tendencies revealed prominent recruitment of pioneers with efficient
light acquisition, short-lived tissues and fast growth (Figure
\ref{fig:Fig4}) \citep{Wright2004, Chave2009b, Herault2011, Reich2014}.
Above an intensity threshold disturbance then additionally involved
short-lived, competitive species, \emph{i.e.} hard pionneers, dominating
the recruitment and reducing its functional diversity.

Following this first phase, the recruitment progressively incorporated
true recruits, \emph{i.e.} trees germinated from the seed bank after
disturbance. The resulting trajectories then corresponded to the
interplay between deterministic processes involving selection and
similarity limitation and random demographic processes of mature forests
progressively emerging again. The balance between both processes
resulted in different trajectories according to the disturbance
intensity.

After low disturbance intensity (T1 plots) the recruitment trajectories
were determined by selective pressures towards light demanding species
that underwent similarity limitation enhancing their functional
diversity. Although the taxonomic composition of the recruitment
resembled the pre-disturbance composition the pool of recruited species
was more restricted and evenly distributed. These restrictions revealed
selective pressures favouring pioneers and light demanding species with
efficient resource acquisition (high SLA and leaf chlorophyll content)
and inexpensive, short-lived tissues (low leaf thickness and thoughness,
small Hmax and low wood density and bark thickness). In parallel the
recruitment's functional diversity increased, equating or exceeding this
of control plots, revealing an overdispersion of functional traits
driven by the limitation of similarity. At this disturbance intensity,
recruitment seemed preserved from the competitive exclusion of hard
pioneers which would have prevented the maintenance of inferior
competitors in the community and would thus have lowered the functional
diversity \citep{Hubbell1999, Sheil2003, Bongers2009}. The low dominance
of hard pioneers might result from recruitment and dispersal limitation
due to the short dispersal distance observed for tropical trees,
specifically in Paracou with the genetic clumping of some pioneers
\citep{Leclerc2015, Scotti2015a}. This deterministic processes, along
with the enhanced recruitment diversity, supported the Intermediate
Disturbance Hypothesis and advocated the role of low intensity
disturbnce to maintain forests high diversity despite the stochastic
recruitment of mature forests \citep{Molino2001, Sheil2003}.

The trajectories after intense disturbance, first driven by the
settlement of hard pioneers, progressively matched the same progressive
balance between deterministic and stochastic processes. This translated
by a progressive decrease of recruitment taxonomic turnover and an
increase of functional diversity. Although acquisitive strategies
remained dominant (high leaf chlorophyll content and low wood density
and leaf toughness), the weighted values of other traits stabilized and
the SLA and bark thickness decreased again (then following a unimodal
trajectory with a peak after the first recruitement phase). The 15 years
laps of the unimodal turnover and traits trajectories corresponding to
the first recruitment phase matched the life expectancy of hard pionners
and of their competitive pressure. The recruitment would progressively
shift towards long-lived pionneers which participate to forest recovery
as they might have been part of pre-disturbance communities, but still
hold dominant more acquisitive functional strategies.

\subsection{On the resilience of the recruitment
process}\label{on-the-resilience-of-the-recruitment-process}

For all plots the richness and functional diversity of recruitment had
recovered thirty years after disturbance to levels equivalent as those
observed short after disturbance and in control plots. In contrast the
species distribution evenness and functional diversity of recruited
trees remained lower than initial and control values, revealing similar
but unachieved trajectories as those of other plots.

Still, for all treatments recruitment processes were restored, matching
the stochastic recruitment of mature forests and showing low taxonomic
turnover between recruited trees and pre-disturbance communities.
Communities trajectories would then maintain the initial differences in
taxonomic composition among communities \citep{Fukami2005}. Similarly
the functional traits trajectories proved very different among plots and
treatments, sometimes even showing opposed tendencies (like for the Leaf
Chlorophyll Content, SLA or Hmax). More than commonly though, then, the
taxonomic and functional recovery of communities would depend on the
pre-disturbance ecosystem characteristics
\citep{Anderson2007, Herault2018}. This makes these commmunities both
functionally and taxonomically resilient despite the settlment of long
lived pioneers which make it a long term process for high disturbance
intensity as evidenced by the persistent impact on traits trajectories.
Our results extend previous ones from the Paracou experiment, 10 years
\citep{Molino2001} and 20 years \citep{Baraloto2012a} after disturbance
which already suggested the resilience of taxonomic and functional
composition.

These conclusions however only hold for a single disturbance event,
given the involvement of the seed bank trigerring a storage effect
likely to modify the resilience of the community and the trajectories
after other disturbance event. In this hypothetical case, the
competitive exclusion among dormant life-stage (seeds or even seedlings)
would be harsher and likely bring more radical changes in the
recruitment composition and functional profile of the community.

\section{Conclusion}\label{conclusion}

The 30 years monitoring of Paracou plots highlighted contrasting
recruitment trajectories determined by the disturbance intensity. In the
short-term communities response was driven by the enhanced growth of
grown saplings benefiting from the alleviated competition and the
environmental changes. Above an intensity threshold the recruitment was
besides dominated by hard-pioneers radically changing the recruitment
composition, diversity and, likely, functioning. In the long-term
response was driven by recruits from the seed bank which underwent
dselection towards light demanding species and similarity limitation
enhancing the functional diversity. These deterministic processes
followed a gradual balance with the stochastic recruitment of mature
forests which eventually restored communities diversity and composition,
maintaining their initial differences. Although forest communities
proved resilient to intense disturbance this appeared to be a long-term
processes likely only valid for single disturbance events.

\begin{center}\rule{0.5\linewidth}{\linethickness}\end{center}

%----------------------------------------------------------------------------------------
%	REFERENCE LIST
%----------------------------------------------------------------------------------------

\bibliographystyle{mee}
\bibliography{references.bib}

%----------------------------------------------------------------------------------------

\end{document}
