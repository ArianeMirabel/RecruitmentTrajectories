%%%%%%%%%%%%%%%%%%%%%%%%%%%%%%%%%%%%%%%%%
% Article EcoFoG
% Version 2.1 (23/10/2017)
%
% adapté de :
% Stylish Article
% LaTeX Template
% Version 1.0 (31/1/13)
%
% This template has been downloaded from:
% http://www.LaTeXTemplates.com
%
% Original author:
% Mathias Legrand (legrand.mathias@gmail.com)
%
% License:
% CC BY-NC-SA 3.0 (http://creativecommons.org/licenses/by-nc-sa/3.0/)
%
%%%%%%%%%%%%%%%%%%%%%%%%%%%%%%%%%%%%%%%%%


%----------------------------------------------------------------------------------------
%	PACKAGES AND OTHER DOCUMENT CONFIGURATIONS
%----------------------------------------------------------------------------------------

\documentclass[fleqn,10pt]{ArtEcoFoG} % Document font size and equations flushed left

\setcounter{tocdepth}{3} % Show only three levels in the table of contents section: sections, subsections and subsubsections


% Pandoc environments
\usepackage{framed}
\usepackage{fancyvrb}
\providecommand{\tightlist}{%
  \setlength{\itemsep}{0pt}\setlength{\parskip}{0pt}}
\newcommand{\VerbBar}{|}
\newcommand{\VERB}{\Verb[commandchars=\\\{\}]}
\DefineVerbatimEnvironment{Highlighting}{Verbatim}{commandchars=\\\{\}, fontsize=\scriptsize} % Code R
\definecolor{shadecolor}{RGB}{248,248,248}
\newenvironment{Shaded}{\begin{snugshade}}{\end{snugshade}}
\newcommand{\KeywordTok}[1]{\textcolor[rgb]{0.13,0.29,0.53}{\textbf{{#1}}}}
\newcommand{\DataTypeTok}[1]{\textcolor[rgb]{0.13,0.29,0.53}{{#1}}}
\newcommand{\DecValTok}[1]{\textcolor[rgb]{0.00,0.00,0.81}{{#1}}}
\newcommand{\BaseNTok}[1]{\textcolor[rgb]{0.00,0.00,0.81}{{#1}}}
\newcommand{\FloatTok}[1]{\textcolor[rgb]{0.00,0.00,0.81}{{#1}}}
\newcommand{\ConstantTok}[1]{\textcolor[rgb]{0.00,0.00,0.00}{{#1}}}
\newcommand{\CharTok}[1]{\textcolor[rgb]{0.31,0.60,0.02}{{#1}}}
\newcommand{\SpecialCharTok}[1]{\textcolor[rgb]{0.00,0.00,0.00}{{#1}}}
\newcommand{\StringTok}[1]{\textcolor[rgb]{0.31,0.60,0.02}{{#1}}}
\newcommand{\VerbatimStringTok}[1]{\textcolor[rgb]{0.31,0.60,0.02}{{#1}}}
\newcommand{\SpecialStringTok}[1]{\textcolor[rgb]{0.31,0.60,0.02}{{#1}}}
\newcommand{\ImportTok}[1]{{#1}}
\newcommand{\CommentTok}[1]{\textcolor[rgb]{0.56,0.35,0.01}{\textit{{#1}}}}
\newcommand{\DocumentationTok}[1]{\textcolor[rgb]{0.56,0.35,0.01}{\textbf{\textit{{#1}}}}}
\newcommand{\AnnotationTok}[1]{\textcolor[rgb]{0.56,0.35,0.01}{\textbf{\textit{{#1}}}}}
\newcommand{\CommentVarTok}[1]{\textcolor[rgb]{0.56,0.35,0.01}{\textbf{\textit{{#1}}}}}
\newcommand{\OtherTok}[1]{\textcolor[rgb]{0.56,0.35,0.01}{{#1}}}
\newcommand{\FunctionTok}[1]{\textcolor[rgb]{0.00,0.00,0.00}{{#1}}}
\newcommand{\VariableTok}[1]{\textcolor[rgb]{0.00,0.00,0.00}{{#1}}}
\newcommand{\ControlFlowTok}[1]{\textcolor[rgb]{0.13,0.29,0.53}{\textbf{{#1}}}}
\newcommand{\OperatorTok}[1]{\textcolor[rgb]{0.81,0.36,0.00}{\textbf{{#1}}}}
\newcommand{\BuiltInTok}[1]{{#1}}
\newcommand{\ExtensionTok}[1]{{#1}}
\newcommand{\PreprocessorTok}[1]{\textcolor[rgb]{0.56,0.35,0.01}{\textit{{#1}}}}
\newcommand{\AttributeTok}[1]{\textcolor[rgb]{0.77,0.63,0.00}{{#1}}}
\newcommand{\RegionMarkerTok}[1]{{#1}}
\newcommand{\InformationTok}[1]{\textcolor[rgb]{0.56,0.35,0.01}{\textbf{\textit{{#1}}}}}
\newcommand{\WarningTok}[1]{\textcolor[rgb]{0.56,0.35,0.01}{\textbf{\textit{{#1}}}}}
\newcommand{\AlertTok}[1]{\textcolor[rgb]{0.94,0.16,0.16}{{#1}}}
\newcommand{\ErrorTok}[1]{\textcolor[rgb]{0.64,0.00,0.00}{\textbf{{#1}}}}
\newcommand{\NormalTok}[1]{{#1}}
\usepackage{longtable,booktabs}
\usepackage{caption}
% These lines are needed to make table captions work with longtable:
\makeatletter
\def\fnum@table{\tablename~\thetable}
\makeatother
% longtable 2 columns
% https://tex.stackexchange.com/questions/161431/how-to-solve-longtable-is-not-in-1-column-mode-error
\makeatletter
\let\oldlt\longtable
\let\endoldlt\endlongtable
\def\longtable{\@ifnextchar[\longtable@i \longtable@ii}
\def\longtable@i[#1]{\begin{figure}[t]
\onecolumn
\begin{minipage}{0.5\textwidth}\scriptsize
\oldlt[#1]
}
\def\longtable@ii{\begin{figure}[t]
\onecolumn
\begin{minipage}{0.5\textwidth}\scriptsize
\oldlt
}
\def\endlongtable{\endoldlt
\end{minipage}
\twocolumn
\end{figure}}
\makeatother

\usepackage{graphicx,grffile}
\makeatletter
\def\maxwidth{\ifdim\Gin@nat@width>\linewidth\linewidth\else\Gin@nat@width\fi}
\def\maxheight{\ifdim\Gin@nat@height>\textheight0.8\textheight\else\Gin@nat@height\fi}
\makeatother
% Scale images if necessary, so that they will not overflow the page
% margins by default, and it is still possible to overwrite the defaults
% using explicit options in \includegraphics[width, height, ...]{}
\setkeys{Gin}{width=\maxwidth,height=\maxheight,keepaspectratio}

% User-adder preamble
\usepackage{textcomp} \DeclareUnicodeCharacter{B0}{\textdegree}
\usepackage{tabu}
\renewenvironment{table}{\begin{table*}}{\end{table*}\ignorespacesafterend}
\hyphenation{bio-di-ver-si-ty sap-lings}

%----------------------------------------------------------------------------------------
%	ARTICLE INFORMATION
%----------------------------------------------------------------------------------------

\JournalInfo{Hal 00679993} % Journal information
\Archive{DOI xxxx} % Additional notes (e.g. copyright, DOI, review/research article)

\PaperTitle{30 Years of Recruitment in Tropical Forest After Selective Logging} % Article title

\Authors{
Ariane MIRABEL\textsuperscript{1*}\\ Eric MARCON\textsuperscript{1}\\ Bruno HERAULT\textsuperscript{2}
} % Authors
\affiliation{
\textsuperscript{1}UMR EcoFoG, AgroParistech, CNRS, Cirad, INRA, Université des Antilles,
Université de Guyane.\\ \hspace{1em} Campus Agronomique, 97310 Kourou, France.\\\textsuperscript{2}INPHB (Institut National Ploytechnique Félix Houphoüet Boigny)\\ \hspace{1em} Yamoussoukro, Ivory Coast
}
\affiliation{*\textbf{Contact}: ariane.mirabel@ecofog.gf, http://www.ecofog.gf/spip.php?article47} % Corresponding author

\Keywords{mot-clés, séparés par des virgules} % Keywords - if you don't want any simply remove all the text between the curly brackets
\newcommand{\keywordname}{Mots-clés} % Defines the keywords heading name

%----------------------------------------------------------------------------------------
%	ABSTRACT
%----------------------------------------------------------------------------------------

\Abstract{
Résumé de l'article.
}

%----------------------------------------------------------------------------------------

\begin{document}

\selectlanguage{english}

\flushbottom % Makes all text pages the same height

\maketitle % Print the title and abstract box

\tableofcontents % Print the contents section

\thispagestyle{empty} % Removes page numbering from the first page

%----------------------------------------------------------------------------------------
%	ARTICLE CONTENTS
%----------------------------------------------------------------------------------------


\begin{verbatim}
## Warning: package 'kableExtra' was built under R version 3.3.3
\end{verbatim}

\section{Introduction}\label{introduction}

Determining the response of tropical forests to disturbance is a key to
predict their fate in a global change context. A vast literature has
successfully modeled the response of tropical forest dynamics, carbon
stocks and fluxes to anthropogenic and natural disturbances
\citep{Gourlet-Fleury2000, Putz2012, Martin2015, Piponiot2016}.
Regarding diversity, however, similar attempts have been hindered by
both the huge biological diversity and the scarcity of long-term
monitoring. If the response to disturbance has been identified for
common species assemblages, it usually remained confined to few
commercial and valuable species
\citep{Sebbenn2008, Rozendaal2010, Vinson2015}. Forest dynamics, though,
result from the constantly evolving interactions and feedbacks among
trees and their environment and could therefore only be assessed through
a complete community-scale approach \citep{DeAvila2016}.

Key to understand communities response to disturbance is to identify the
processes shaping the composition and diversity of recruited trees.
Forests dynamics stem from the suit of recruitment process from seed
production, dispersion and germination to seedlings' and saplings'
growth until the adult stage. Recruitment mechanisms result from the
interplay of deterministic environmental processes, like the exclusion
of stress-intolerant species or the limitation of similarity through
resource competition \citep{Ackerly2003, McGill2006}, and stochastic
processes like random dispersal, recruitment and death
\citep{Hubbell2001}. The deterministic processes are inherently linked
to disturbance regime which locally changes ecosystem's biotic and
abiotic conditions and maintains species able of efficient acquisition
of resource but living shortly and poorly resistant to hazards and
diseases \citep{Denslow1980}. They rely on the Intermediate Disturbance
Hypothesis (IDH) that explains the maintenance of tropical forests
biodiversity by the patchy variability of environmental conditions in
space and time \citep{Guitet2018}. Specifically, in tropical wet forests
changes light availability has a central role enhancing the recruitment
of pioneers and light-demanding species after disturbance compared to
mature stands where more competitive shade bearers dominate. Disturbance
then enlarges the ecological range of species in the community
\citep{Molino2001, Bongers2009} and shapes their taxonomic diversity,
vegetative structure, physiology as well as carbon, nutrients, and water
cycles \citep{Anderson-Teixeira2013}. \textgreater{}\textgreater{}(on
parle plus vraiment de IDH maintenant) Empirical tests of the IDH in
tropical rainforests, though, proved hard to succeed and yielded
controversial results \citep{Hubbell1999, Molino2001, Sheil2003}.

During post-disturbance times, the shift from resource-acquisitive to
resource-conservative ecological strategies may be detected in leaves
(leaf thickness, toughness, chlorophyll content and specific area) and
stem (wood specific gravity and bark thickness) and life-history traits
(maximum height at adult stage and class of seed mass)
\citep{Wright2004, Chave2009b, Herault2011}. The relative importance of
recruitment of new individuals and of mortality of distrubance survivors
will shape the new forest and its functioning. Given that disturbance
survivors largely mirror the pre-disturbance forest composition
\citep{Piponiot2018}, predicting the recruitment composition and
diversity trajectories would be a major step towards the prediction of
the future of tropical forest in a changing global environment where
disturbance are expected to become more and more frequent. This would
give insights into the resilience of this hyperdiverse ecosystems,
elucidate the determinism, or not, of tropical forests trajectories,
test the convergence after disturbance of taxonomic and functional
communities towards initial state and also help future adaptative
conservation strategies \citep{Diaz2005, Gardner2007, Schwartz2017}.

In this paper we follow the fate of a recruited tree communities (60121
individuals) over 30 years on a disturbance gradient, with 10 to 60\% of
forest biomass removed. We assessed the taxonomic as well as functional
diversity of recruited trees, using a large functional trait database
covering, the leaf, wood and life-history spectra. We aimed to (i)
assess the role of environmental filtering selecting the recruited trees
according to their competitivity for resource acquisition, (ii) resolve
the convergence of communities and the maintenance of taxonomic
composition in the long term, and (iii) determine the global resilience
of the ecosystem.

\section{Material and Methods}\label{material-and-methods}

\subsection{Study Site}\label{study-site}

The Paracou station is located in a lowland tropical rain forest in
French Guiana (5°18'N and 52°53'W). Climate is tropical wet with mean
annual precipitation averaging 2980 mm.y\textsuperscript{-1} (30-y
period) and a 3-months dry season (\textless{} 100
mm.months\textsuperscript{-1}) from mid-August to mid-November, and a
one-month dry season in March \citep{Wagner2011}. Elevation ranges from
5 to 50 m and mean annual temperature is 26°C. Soils are thin acrisols
over a layer of transformed saprolite with low permeability generating
lateral drainage during heavy rains. The disturbance experiment spread
over a network of twelve 6.25ha plots (Table \ref{tab:Tab1}) that
underwent three disturbance treatments in 1986-1987 \citep{Herault2018}.
Dominant families are Fabaceae, Chrysobalanaceae, Lecythidaceae and
Sapotaceae.

\begin{table}

\caption{\label{tab:Tab1}Intervention table, summary of the disturbance intensity for the 4 plot treatments in Paracou.}
\centering
\begin{tabu} to \linewidth {>{\raggedright}X>{\raggedright}X>{\raggedright}X>{\raggedright}X>{\raggedright}X}
\toprule
Treatment & Timber & Thinning & Fuelwood & \%AGB lost\\
\midrule
Control &  &  &  & 0\\
T1 & DBH $\geq$ 50 cm, commercial species, $\approx$ 10 trees/ha &  &  & $[12\%-33\%]$\\
T2 & DBH $\geq$ 50 cm, commercial species, $\approx$ 10 trees/ha & DBH $\geq$ 40 cm, non-valuable species, $\approx$ 30 trees/ha &  & $[33\%-56\%]$\\
T3 & DBH $\geq$ 50 cm, commercial species, $\approx$ 10 trees/ha & DBH $\geq$ 50 cm, non-valuable species, $\approx$ 15 trees/ha & 40 cm $\leq$ DBH $\leq$ 50 cm, non-valuable species, $\approx$ 15 trees/ha & $[35\%-56\%]$\\
\bottomrule
\end{tabu}
\end{table}

\subsection{Inventories Protocol and Dataset
Collection}\label{inventories-protocol-and-dataset-collection}

All trees above 10 cm DBH were mapped and measured annually since 1984.
During inventories, trees were first identified with a vernacular name
assigned by the field team, and afterward with a scientific name
assigned by a botanist during regular botanical campaigns. Botanical
campaigns have been carried out every 5 to 6 years from 2003 onwards.
This raised methodological issues as vernacular names usually correspond
to different botanical species, resulting in significant taxonomic
uncertainties that were propagated to composition and diversity metrics.
Vernacular names were replaced through multinomial trials
\(M_v\Big(\big[s_1, s_2, …, s_N\big],\big[\alpha_1, \alpha_2,…, \alpha_3\big]\Big)\)
based on the observed association probability
\(\big[\alpha_1, \alpha_2,…, \alpha_3\big]\) between each vernacular
name \emph{v} and the species \(\big[s_1, s_2, …, s_N\big]\) recorded in
the inventory. See appendix 1 and \citet{Aubry-Kientz2013} for the
detailed methodology. To avoid remaining identification caveats, the
simulated botanical inventories were reported at genus level.

Six functional traits, representing leaf economics (leaves thickness,
toughness, total chlorophyll content and specific leaf area, the leaf
area per unit dry mass), wood economics (wood specific gravity and bark
thickness) and life history traits (maximum specific height and seed
mass), come from the BRIDGE project \footnote{http://www.ecofog.gf/Bridge/}
where trait values were measured on nine french guianan forest plots,
including two in Paracou. Missing trait values (X\%) were filled using
multivariate imputation by chained equation (mice) restricted to samples
pertaining to the next higher taxonomic level, in order to account for
the phylogenetic signal of the functional traits (refs MICE). As seed
mass information corresponds to a classification into mass classes, no
data filling process was applied so analysis were performed only
considering the 414 botanical species of the seed mass dataset.

Functional trajectories were estimated with the Rao quadratic entropy
using community weighted means (CWM) \citep{Diaz2007, Garnier2004}. Seed
mass trajectories were reported by the proportion of each class recorded
in the inventories. All composition and diversity metrics are the
average obtained after 50 iterations of uncertainty propagation
(taxonomy and trait values).

\subsection{Recruitment trajectories}\label{recruitment-trajectories}

We split the forest community in `survivors, i.e.~trees that survived
the disturbance, and post-disturbance recruited trees. Two recruitment
metrics were examined: on the one hand the ``punctual recruitment'' by
2-year intervals after disturbance, on the other hand all recruited
trees since disturbance, hereafter ``accumulated recruits''. The
taxonomic diversity was assessed through Richness and the Hill number
translation of Shannon and Simpson indices
\citep{Hill1973, chao2015estimating, Marcon2015b}. These three indices
belong to the set of HCDT or generalized entropy, respectively
corresponding to the 0, 1 and 2 order of diversity (\emph{q}), which
grasps the balance between richness and evenness in the community, with
common species weighting more than rare ones when \emph{q} increases.
The similarity between the recruited trees and the pre-disturbance
forest was measured with the turnover metrics detailed in
\citet{Podani2013a}. To determine whether recruitment trajectories
ensued from a pure random process, observed trajectories were compared
to those generated by 50 repetitions of a random null model shuffling
individuals among plots while preserving species abundance and plots'
tree density.

To draw plots trajectories we applied a moving average with a one step
window allowing to mitigate the heterogeneity of inventory protocols
between years.

\section{Results}\label{results}

\subsection{Recruitment Diversity}\label{recruitment-diversity}

\subsubsection{Taxonomic diversity}\label{taxonomic-diversity}

Punctual recruits' diversity followed a consistent trajectory among
disturbance treatments with first higher richness and lower evenness
than in control plots and then equivalent richness and lower evenness
(Figure (\ref{fig:Fig1}). The richness of accumulated recruits was
higher for the disturbed plots, consistently with the increase of
recruited trees following disturbance, but their evenness was equivalent
or lower, specifically for the most disturbed plots.

\begin{figure*}

{\centering \includegraphics[width=0.8\linewidth]{RecruitmentTrajectories_files/figure-latex/Fig1-1} 

}

\caption{Trajectories of Richness, Shannon and Simpson diversity for 2-years laps punctual  recruitment (upper panels) and divergence to null model (lower panels). Lines colors refer to the perturbation regime: green for control, blue for T1, orange for T2 and red for T3 disturbance treatments. Plain lines correspond to the median observed after uncertainty propagation and are given along with the 95\% confidence interval (grey envelope).}\label{fig:Fig1}
\end{figure*}

Punctual and accumulated recruitment diversity of orders 0, 1 and 2 were
then compared to a null random recruitment model. In control plots,
punctual recruitment values remained equivalent or higher than with the
null model (Figure \ref{fig:Fig1}) while in disturbed plots punctual
recruitment richness and evenness remained lower than that of the null
model whenever the disturbance treatment. Accumulated recruitment
richness (order 0) and evenness (order 2) were higher or equivalent to
those of the null model for plots T1 and some plots T2 but lower for
plots T3 and a plot T2 (Figure A1, Appendix I).

\subsubsection{Functional diversity}\label{functional-diversity}

\begin{figure}

{\centering \includegraphics{RecruitmentTrajectories_files/figure-latex/Fig2-1} 

}

\caption{Functional diversity of punctual recruited trees from the considered functional traits and divergence to null model. Values reported correspond to the plot-level median and the 95\% confidence interval obtained after 50 repetition of the taxonomic uncertainty propagation and the functional database gap-filling processes and 50 run of the null model. Lines colors correspond to the logging treatment initially applied (green for control, blue for T1,orange for T2 and red for T3).}\label{fig:Fig2}
\end{figure}

Recruits functional diversity lower after disturbance for 15 to 20 years
period before increasing to a similar or higher value as observed for
control plots (Figure \ref{fig:Fig3}). Trajectories of recruited trees
in the functional spaces showed the dominance after disturbance of
species displaying large exchange surface area and light tissues (high
SLA, low leaf toughness and low wood specific gravity) (Figure
\ref{fig:Fig3}). All traits trajectories displayed univariate CWM
trajectories, except SLA and leaf thickness that displayed a unimodal
trajectory.

\begin{figure*}

{\centering \includegraphics[width=0.8\linewidth]{RecruitmentTrajectories_files/figure-latex/Fig3-1} 

}

\caption{Community weighted means (CWM) of the four disturbance treatment for the four leaf traits, the two stem traits  and the specific Hmax. Values reported correspond to the plot-level median obtained after 50 repetition of the taxonomic uncertainty propagation and the functional database gap-filling processes. Lines colors correspond to the disturbance intensity (green for control, blue for T1,orange for T2 and red for T3).}\label{fig:Fig3}
\end{figure*}

\subsection{Recruitment Turnover}\label{recruitment-turnover}

In control plots, species turnover remained highly stable for the 30
sampled years (Figure \ref{fig:Fig4}). In disturbed plots, turnover
displayed a unimodal response to disturbance, with maximum reached
around 15 years and with a value positively correlated to the
disturbance intensity (\(\rho_{spearman}=0.93\)). The turnover
trajectory returned close to zero for all plots 30 years after
disturbance.

\begin{figure}

{\centering \includegraphics{RecruitmentTrajectories_files/figure-latex/Fig4-1} 

}

\caption{Trajectories over the 30 sampled years of the abundance-based turnover between recruited trees and intial communities before disturbance. Grey envelopes correspond to the 0.025 and 0.975 percentiles of the uncertainty propagation procedue and lines to the median in green for control, blue for T1,orange for T2 and red for T3).}\label{fig:Fig4}
\end{figure}

\section{Discussion}\label{discussion}

From the 30 years of forest dynamics survey in the Paracou station, we
highlighted contrasting recruitment patterns depending on disturbance
intensity. Disturbance increased the recruitment rate which subsequently
increased the richness of recruitment but impaired its evenness, all the
more so that disturbance intensity was high. After disturbance the
recruitment was dominated by a restricted pool of species and Shannon
and Simpson diversities decreased down to lower values than those of
control plots.

\subsection{On the underlyings of the hump-shaped
trajectories}\label{on-the-underlyings-of-the-hump-shaped-trajectories}

The punctual recruitment richness, some key functional traits (SLA and
bark thickness) and the species turn-over trajectories exhibited
hump-shaped curves so that, for the 10-15 first years following
disturbance, recruitment processes seemed driven by the growth of
pre-disturbance saplings benefitting from the environmental changes and
alleviated competition \citep{Herault2010}. After this recruitment
progressively incorporated true recruits, \emph{i.e.} individual trees
that had germinated after the disturbance, undergoing competition again
and environmental selection. This translated first by a stable
functional diversity of the recruitment, equivalent as this of control
plots, and low difference to random null model, at least for the lower
intensities. According to the SLA trajectory (Figure \ref{fig:Fig4}),
these true recruits dominating the recruitment 12 years after
disturbance are mainly pionneer species.

After this first recruitment phase, different trajectories were observed
according to the disturbance intensity. In the case of low disturbance
intensity (T1 plots) the recruitment resembled pre-disturbance species
composition. Besides it displayed on the one hand decreasing eveness
compared to control plots and random models, translating selective
recruitment, and on the other hand increasing functional diversity
reaching the levels of control plots or above, translating some
competitive exclusion. The second recruitment phase at this level of
disturbance is then dominated by pioneers and light demanding species
either competing with each other and resulting in a functionally
diversified community, or preserved from the overwhelming settlement of
hard pioneers by recruitment limitation
\citep{Hubbell1999, Sheil2003, Bongers2009}. Recruitment limitation of
hard pioneers, mitigating competitive exclusion and maintaining some
inferior competitiors in the community, would be consistent with the
short dispersal distance observed for tropical trees, specifically in
Paracou with the genetic clumping of some pioneers
\citep{Leclerc2015, Scotti2015a}. In the case of high disturbance
intensity, the species turnover remains very high after the initial 10
year phasis and the functional diversity decreased sharply towards a
restricted range of acquisitive functional strategies.\\
Both recruitment phase were marked for nearly all functional traits of
the leaf and stem and for life history traits, reflecting the dominance
of high resources acquisition strategies
\citep{Wright2004, Chave2009b, Herault2011, Reich2014}. However the
first recruitment phase corresponded to a sharp increase in SLA and bark
thickness, reflecting the dominance of short-lived pionneers, while the
second recruitment phase corresponded to a decrease in leaf thoughness,
Wood Specific Gravity and Hmax relecting the incorporation of long-lived
pioneers.

\subsection{On the resilience of the recruitment
process}\label{on-the-resilience-of-the-recruitment-process}

Thirty years after disturbance, recruitment functional diversity and
richness had recovered equivalent levels as those observed for the plots
short after after disturbance and equivalent to those of control plots.
The functional and taxonomic composition of the recruitment and the
evenness of recruited species distribution, though, remained lower for
all disturbed plots, along with and weighted means of all functional
traits was remained altered thirty years after disturbance. Would this
mean that ecosystem functioning recover faster than taxonomic
composition? The link between functional traits and ecosystem
functionning is not trivial but our results may highlight the key role
of functional redundancy in ecological systems and also suggest.
\textgreater{}\textgreater{} that the recovery trajectory may be, more
than commonly thought, upon dependence of the pre-disturbance functional
ecosystem signature \citep{Herault2018}. -\textgreater{} What makes us
say that?

Whatever the disturbance intensity, the recruitment turnover compared to
initial stand ended up close to zero. This argued for the high
dependence of the composition recovery trajectory on the pre-disturbance
ecosystem characteristics \citep{Anderson2007}. In other words, initial
compositional variation caused tree communities to remain divergent in
taxonomic composition, even though these same tree communities strongly
converged in functional space \citep{Fukami2005}. This makes these
commmunities both functionally and taxonomically resilient despite the
settlment of long lived pioneers which make it a long term process for
high disturbance intensity. \textgreater{}\textgreater{} Not that much
because the long lived pioneers probably remain long after disturbance,
it would be in the long term.

Our results extend previous ones from the Paracou experiment, 10 years
\citep{Molino2001} and 20 years \citep{Baraloto2012a} after disturbance
that suggested the recovery towards pre-disturbance taxonomic and
functional composition. This is however only valid for a single
disturbance event, given that second recruitment phase assumed to rely
on the seed bank triggers a storage effect likely to modify the
recruitment trajectories in case of a new disturbance event. In this
hypothetical case, the competitive exclusion among dormant life-stage
(seeds or even seedlings) would be harsher and would bring more radical
changes in the recruitment composition and functional profile.

\section{Conclusion}\label{conclusion}

Our long-term study of diversity trajectories after disturbance
disentangled the mechanisms underlying the forest trajectory in the
diversity space. While, in undisturbed forests, mechanisms like negative
density dependence enhanced species diversity, disturbance induced new
mechanisms depending on its intensity. In the short-term communities
response was always driven by the enhanced growth of grown saplings
benefitting from alleviated competition and low selective pressure
resulting in increased communities taxonomic and functional evenness. In
the long-term communities trajectory strongly depended on the
disturbance intensity. At low intensity competition resumed to shape a
functionally diversified and even community preserved from the
over-whelming hard pioneers by recruitment limitation. For high
intensity, strong environmental filters maintained high species turnover
and a recruitment restricted to long lived pionneers. Communities
composition and functioning proved consistently resilient to single
disturbance events but the functional redundancy and the diversified
seed bank of mature forests proved altered and consequently communities
resilience themselves.

\begin{center}\rule{0.5\linewidth}{\linethickness}\end{center}

%----------------------------------------------------------------------------------------
%	REFERENCE LIST
%----------------------------------------------------------------------------------------

\bibliographystyle{mee}
\bibliography{references.bib}

%----------------------------------------------------------------------------------------

\end{document}
