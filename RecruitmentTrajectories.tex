%%%%%%%%%%%%%%%%%%%%%%%%%%%%%%%%%%%%%%%%%
% Article EcoFoG
% Version 2.1 (23/10/2017)
%
% adapté de :
% Stylish Article
% LaTeX Template
% Version 1.0 (31/1/13)
%
% This template has been downloaded from:
% http://www.LaTeXTemplates.com
%
% Original author:
% Mathias Legrand (legrand.mathias@gmail.com)
%
% License:
% CC BY-NC-SA 3.0 (http://creativecommons.org/licenses/by-nc-sa/3.0/)
%
%%%%%%%%%%%%%%%%%%%%%%%%%%%%%%%%%%%%%%%%%


%----------------------------------------------------------------------------------------
%	PACKAGES AND OTHER DOCUMENT CONFIGURATIONS
%----------------------------------------------------------------------------------------

\documentclass[fleqn,10pt]{ArtEcoFoG} % Document font size and equations flushed left

\setcounter{tocdepth}{3} % Show only three levels in the table of contents section: sections, subsections and subsubsections


% Pandoc environments
\usepackage{framed}
\usepackage{fancyvrb}
\providecommand{\tightlist}{%
  \setlength{\itemsep}{0pt}\setlength{\parskip}{0pt}}
\newcommand{\VerbBar}{|}
\newcommand{\VERB}{\Verb[commandchars=\\\{\}]}
\DefineVerbatimEnvironment{Highlighting}{Verbatim}{commandchars=\\\{\}, fontsize=\scriptsize} % Code R
\definecolor{shadecolor}{RGB}{248,248,248}
\newenvironment{Shaded}{\begin{snugshade}}{\end{snugshade}}
\newcommand{\KeywordTok}[1]{\textcolor[rgb]{0.13,0.29,0.53}{\textbf{{#1}}}}
\newcommand{\DataTypeTok}[1]{\textcolor[rgb]{0.13,0.29,0.53}{{#1}}}
\newcommand{\DecValTok}[1]{\textcolor[rgb]{0.00,0.00,0.81}{{#1}}}
\newcommand{\BaseNTok}[1]{\textcolor[rgb]{0.00,0.00,0.81}{{#1}}}
\newcommand{\FloatTok}[1]{\textcolor[rgb]{0.00,0.00,0.81}{{#1}}}
\newcommand{\ConstantTok}[1]{\textcolor[rgb]{0.00,0.00,0.00}{{#1}}}
\newcommand{\CharTok}[1]{\textcolor[rgb]{0.31,0.60,0.02}{{#1}}}
\newcommand{\SpecialCharTok}[1]{\textcolor[rgb]{0.00,0.00,0.00}{{#1}}}
\newcommand{\StringTok}[1]{\textcolor[rgb]{0.31,0.60,0.02}{{#1}}}
\newcommand{\VerbatimStringTok}[1]{\textcolor[rgb]{0.31,0.60,0.02}{{#1}}}
\newcommand{\SpecialStringTok}[1]{\textcolor[rgb]{0.31,0.60,0.02}{{#1}}}
\newcommand{\ImportTok}[1]{{#1}}
\newcommand{\CommentTok}[1]{\textcolor[rgb]{0.56,0.35,0.01}{\textit{{#1}}}}
\newcommand{\DocumentationTok}[1]{\textcolor[rgb]{0.56,0.35,0.01}{\textbf{\textit{{#1}}}}}
\newcommand{\AnnotationTok}[1]{\textcolor[rgb]{0.56,0.35,0.01}{\textbf{\textit{{#1}}}}}
\newcommand{\CommentVarTok}[1]{\textcolor[rgb]{0.56,0.35,0.01}{\textbf{\textit{{#1}}}}}
\newcommand{\OtherTok}[1]{\textcolor[rgb]{0.56,0.35,0.01}{{#1}}}
\newcommand{\FunctionTok}[1]{\textcolor[rgb]{0.00,0.00,0.00}{{#1}}}
\newcommand{\VariableTok}[1]{\textcolor[rgb]{0.00,0.00,0.00}{{#1}}}
\newcommand{\ControlFlowTok}[1]{\textcolor[rgb]{0.13,0.29,0.53}{\textbf{{#1}}}}
\newcommand{\OperatorTok}[1]{\textcolor[rgb]{0.81,0.36,0.00}{\textbf{{#1}}}}
\newcommand{\BuiltInTok}[1]{{#1}}
\newcommand{\ExtensionTok}[1]{{#1}}
\newcommand{\PreprocessorTok}[1]{\textcolor[rgb]{0.56,0.35,0.01}{\textit{{#1}}}}
\newcommand{\AttributeTok}[1]{\textcolor[rgb]{0.77,0.63,0.00}{{#1}}}
\newcommand{\RegionMarkerTok}[1]{{#1}}
\newcommand{\InformationTok}[1]{\textcolor[rgb]{0.56,0.35,0.01}{\textbf{\textit{{#1}}}}}
\newcommand{\WarningTok}[1]{\textcolor[rgb]{0.56,0.35,0.01}{\textbf{\textit{{#1}}}}}
\newcommand{\AlertTok}[1]{\textcolor[rgb]{0.94,0.16,0.16}{{#1}}}
\newcommand{\ErrorTok}[1]{\textcolor[rgb]{0.64,0.00,0.00}{\textbf{{#1}}}}
\newcommand{\NormalTok}[1]{{#1}}
\usepackage{longtable,booktabs}
\usepackage{caption}
% These lines are needed to make table captions work with longtable:
\makeatletter
\def\fnum@table{\tablename~\thetable}
\makeatother
% longtable 2 columns
% https://tex.stackexchange.com/questions/161431/how-to-solve-longtable-is-not-in-1-column-mode-error
\makeatletter
\let\oldlt\longtable
\let\endoldlt\endlongtable
\def\longtable{\@ifnextchar[\longtable@i \longtable@ii}
\def\longtable@i[#1]{\begin{figure}[t]
\onecolumn
\begin{minipage}{0.5\textwidth}\scriptsize
\oldlt[#1]
}
\def\longtable@ii{\begin{figure}[t]
\onecolumn
\begin{minipage}{0.5\textwidth}\scriptsize
\oldlt
}
\def\endlongtable{\endoldlt
\end{minipage}
\twocolumn
\end{figure}}
\makeatother

\usepackage{graphicx,grffile}
\makeatletter
\def\maxwidth{\ifdim\Gin@nat@width>\linewidth\linewidth\else\Gin@nat@width\fi}
\def\maxheight{\ifdim\Gin@nat@height>\textheight0.8\textheight\else\Gin@nat@height\fi}
\makeatother
% Scale images if necessary, so that they will not overflow the page
% margins by default, and it is still possible to overwrite the defaults
% using explicit options in \includegraphics[width, height, ...]{}
\setkeys{Gin}{width=\maxwidth,height=\maxheight,keepaspectratio}

% User-adder preamble
\usepackage{textcomp} \DeclareUnicodeCharacter{B0}{\textdegree}
\usepackage{tabu}
\renewenvironment{table}{\begin{table*}}{\end{table*}\ignorespacesafterend}
\hyphenation{bio-di-ver-si-ty sap-lings}

%----------------------------------------------------------------------------------------
%	ARTICLE INFORMATION
%----------------------------------------------------------------------------------------

\JournalInfo{Hal 00679993} % Journal information
\Archive{DOI xxxx} % Additional notes (e.g. copyright, DOI, review/research article)

\PaperTitle{30 Years of Post-disturbance Recruitment in Tropical Forest} % Article title

\Authors{
Ariane MIRABEL\textsuperscript{1*}\\ Eric MARCON\textsuperscript{1}\\ Bruno HERAULT\textsuperscript{2}
} % Authors
\affiliation{
\textsuperscript{1}UMR EcoFoG, AgroParistech, CNRS, Cirad, INRA, Université des Antilles,
Université de Guyane.\\ \hspace{1em} Campus Agronomique, 97310 Kourou, France.\\\textsuperscript{2}INPHB (Institut National Ploytechnique Félix Houphoüet Boigny)\\ \hspace{1em} Yamoussoukro, Ivory Coast
}
\affiliation{*\textbf{Corresponding author}: ariane.mirabel@ecofog.gf, http://www.ecofog.gf/spip.php?article47} % Corresponding author

\Keywords{Taxonomic and Functional Diversity, Recruitment, Resilience, Tropical Forests, Disturbance Dynamics} % Keywords - if you don't want any simply remove all the text between the curly brackets
\newcommand{\keywordname}{Keywords} % Defines the keywords heading name

%----------------------------------------------------------------------------------------
%	ABSTRACT
%----------------------------------------------------------------------------------------

\Abstract{
Résumé de l'article.
}

%----------------------------------------------------------------------------------------

\begin{document}

\selectlanguage{english}

\flushbottom % Makes all text pages the same height

\maketitle % Print the title and abstract box

\tableofcontents % Print the contents section

\thispagestyle{empty} % Removes page numbering from the first page

%----------------------------------------------------------------------------------------
%	ARTICLE CONTENTS
%----------------------------------------------------------------------------------------


\section{Introduction}\label{introduction}

Determining the response of tropical forests to disturbance is a key to
predict their fate in the global changing context. In the last decades,
tropical forests experienced a wide range of disturbance, from radical
land-use changes for agriculture or mining
\citep{Dezecache2017a, Dezecache2017b} to more insidious changes of
communities structure, diversity and functioning following anthropogenic
activities like selective logging \citep{Baraloto2012a, Herault2016} or
climate change \citep{Aubry-Kientz2015}. In that respect a vast
literature successfully modeled communities response to disturbance in
terms of tree growth \citep{Gourlet-Fleury2000}, tree height
\citep{Rutishauser2016}, carbon stocks and fluxes
\citep{Putz2012, Martin2015, Piponiot2016}. Similar approaches regarding
forest composition and diversity, however, have been hindered by the
huge biological diversity, often focusing on common or mainly commercial
species, and the scarcity of long-term monitoring
\citep{Sebbenn2008, Rozendaal2010, Vinson2015}.

Communites trajectories after disturbance, defined here as the evolution
of communities diversity along time, depend on the trees surviving from
before disturbance and on the new trees recruited afterward
\citep{Herault2018}. Surviving trees proved to mirror the composition of
pre-disturbance forest so communities' diversity response is driven by
the diversity and composition of recruited trees, which build the future
community and determine the resilience of pre-disturbance state.
Recruitment trajectories depend first on the composition and diversity
of initial community that partly condition the available pool of
recruited species \citep{Herault2018}. Then trajectories depend on
recruitment processes either stochastic, like random dispersal,
recruitment and death \citep{Hubbell2001}, or deterministic like
niche-based competition processes \citep{Adler2007} While stochastic
processes would build communities similar to random samples of larger
regional metacommunities, deterministic processes rely on the abiotic
environment and filter-out recruited species according to their ecology.
Understand communities response to disturbance comes back first to
estimate the balance between these determinants, the initial composition
and the different recruitment processes, and then to elucidate
communities resilience and the maintenance of the initial differences
among local communities \citep{Diaz2005, Gardner2007, Schwartz2017}.

The processes shaping communities response will differently affect the
functional diversity \citep{Kunstler2016}, that considers species
functioning and ecology \citep{Violle2007}, and the taxonomic diversity,
which consider all species equal. The coupling between functional and
taxonomic trajectories will then be insightful for the identification of
ecological processes at stake, and communities resilience should be
addressed considering both functional and taxonomics characteristics
\citep{Fukami2005}. Functional trajectories of recruited trees are
shaped by the deterministic processes that are oriented towards the use
of limiting resources. Recruitment processes will then either filter-out
the les efficient functional strategies and restrict the functional
diversity, or exclude the less competitive ones and increase the
functional diversity though the competition among species and limitation
of their functional similarity \citep{Ackerly2003, McGill2006}. The
disturbance response of tropical forests where the light is limiting
would then be a shift from slow-growing, long-living species with
``conservative'' resource use to fast growing, resource ``acquisitve''
species \citep{Denslow1980, Molino2001, Bongers2009}. To detect these
processes the large functional trait-based litterature identified key
leaf, wood and life-history functional traits assessing species ecology
and resources acquisition strategy
\citep{Wright2004, Chave2009b, Herault2011}.

In this paper we follow the fate of a recruited tree communities (60121
individuals) over 30 years on a large disturbance gradient, with 10 to
60\% of forest biomass removed. We assess the taxonomic and functional
diversity of recruited trees, using a large functional trait database
covering the leaf, wood and life-history spectra. We compare the
observed trajectories to null models representing random trees
recruitment and randomized functional traits. We aimed (i) to assess the
role of deterministic processes compared to stochastic recruitment after
disturbance, (ii) assess the taxonomic and functional convergence of
forest communities and the maintenance of taxonomic composition in the
long term, and (iii) determine the degree of resilience of the
ecosystem.

\section{Material and Methods}\label{material-and-methods}

\subsection{Study Site}\label{study-site}

The Paracou station is located in a lowland tropical rain forest in
French Guiana (5°18'N and 52°53'W). Climate is tropical wet with mean
annual precipitation averaging 2980 mm.y\textsuperscript{-1} (30-y
period) and a 3-months dry season (\textless{} 100
mm.months\textsuperscript{-1}) from mid-August to mid-November, and a
one-month dry season in March \citep{Wagner2011}. Elevation ranges from
5 to 50 m and mean annual temperature is 26°C. Soils are thin acrisols
over a layer of transformed saprolite with low permeability generating
lateral drainage during heavy rains. The disturbance experiment spread
over a network of twelve 6.25ha plots (Table \ref{tab:Tab1}) that
underwent three disturbance treatments in 1986-1987 \citep{Herault2018}.
Dominant families are Fabaceae, Chrysobalanaceae, Lecythidaceae and
Sapotaceae.

\begin{table}

\caption{\label{tab:Tab1}Intervention table, summary of the disturbance intensity for the 4 plot treatments in Paracou.}
\centering
\begin{tabu} to \linewidth {>{\raggedright}X>{\raggedright}X>{\raggedright}X>{\raggedright}X>{\raggedright}X}
\toprule
Treatment & Timber & Thinning & Fuelwood & \%AGB lost\\
\midrule
Control &  &  &  & 0\\
T1 & DBH $\geq$ 50 cm, commercial species, $\approx$ 10 trees/ha &  &  & $[12\%-33\%]$\\
T2 & DBH $\geq$ 50 cm, commercial species, $\approx$ 10 trees/ha & DBH $\geq$ 40 cm, non-valuable species, $\approx$ 30 trees/ha &  & $[33\%-56\%]$\\
T3 & DBH $\geq$ 50 cm, commercial species, $\approx$ 10 trees/ha & DBH $\geq$ 50 cm, non-valuable species, $\approx$ 15 trees/ha & 40 cm $\leq$ DBH $\leq$ 50 cm, non-valuable species, $\approx$ 15 trees/ha & $[35\%-56\%]$\\
\bottomrule
\end{tabu}
\end{table}

\subsection{Inventories Protocol and Dataset
Collection}\label{inventories-protocol-and-dataset-collection}

All trees above 10 cm DBH were mapped and measured annually since 1984.
During inventories, trees were first identified with a vernacular name
assigned by the field team, and afterward with a scientific name
assigned by a botanist during regular botanical campaigns. Botanical
campaigns have been carried out every 5 to 6 years from 2003 onwards.
These changes in identification protocol raised methodological issues as
vernacular names usually correspond to different botanical species,
resulting in significant taxonomic uncertainties that were propagated to
composition and diversity metrics. Vernacular names were replaced
through multinomial trials
\(M_v\Big(\big[s_1, s_2, …, s_N\big],\big[\alpha_1, \alpha_2,…, \alpha_3\big]\Big)\)
based on the observed association probability
\(\big[\alpha_1, \alpha_2,…, \alpha_3\big]\) between each vernacular
name \emph{v} and the species \(\big[s_1, s_2, …, s_N\big]\) recorded in
the inventory. See appendix 1 and \citet{Aubry-Kientz2013} for the
detailed methodology. To avoid remaining identification caveats, the
simulated botanical inventories were reported at genus level.

Eight functional traits were considered, representing leaf economics
(leaves thickness, toughness, total chlorophyll content and specific
leaf area), wood economics (wood specific gravity and bark thickness)
and life history traits (maximum specific height and seed mass). Traits
were exctracted from the BRIDGE project \footnote{http://www.ecofog.gf/Bridge/}
where trait values were measured on nine forest plots infrench guianan,
including two in Paracou. Missing trait values of the trait database
(10\%) were filled by multivariate imputation by chained equation using
the Mice r package \citep{Mice2011}. As traits variability was lower
within genus and families, we accounted for the phylogenetic signal of
the functional traits by restricting the gap filling processes to
samples pertaining to the next higher taxonomic level. As seed mass
information corresponded to a classification into discrete mass classes,
no data filling process was applied so analysis were performed only
considering the 414 botanical species of the seed mass dataset.

\subsection{Recruitment trajectories}\label{recruitment-trajectories}

To disentangle the recruitment processes from overall dynamics,
communities were split into per-disturbance surviving trees and those
recruited since disturbance. Recruited communities were examined either
considering the ``punctual recruitment'', \emph{i.e.} recruited trees by
2-year intervals, or all recruits since disturbance as the ``accumulated
recruits''. Eventually, in disturbed plots the recruited communities
were examined distinguishing the undisturbed and logging gap areas to
test the validity of recruitment processes for the whole area.

The taxonomic diversity was assessed through Richness and the Hill
number translation of Shannon and Simpson indices
\citep{Hill1973, chao2015estimating, Marcon2015b}.\\
The three diversities belong to the set of HCDT or generalized entropy,
respectively corresponding to the 0, 1 and 2 order of diversity
(\emph{q}), which grasps the balance between richness and evenness in
the community through the value of \emph{q} that emphasizes common
species. Functional trajectories were estimated with the Rao quadratic
entropy and completed by the trajectories of traits community weighted
means (CWM), representing the average trait value in a community
weighted by relative abundance of the species carrying each value
\citep{Diaz2007, Garnier2004}. Seed mass trajectories were reported by
the proportion of each class recorded in the inventories. The similarity
between the recruited trees and the pre-disturbance forest was measured
with the turnover metrics detailed in \citet{Podani2013a}. To estimate
the importance of stochastic processes the recruitment was compared to
the trajectories of a random sampling. For the taxonomic trajectories
the random sampling was a shuffle of trees among plots that preserved
species abundance and tree density, and for the functional diversity it
was a shuffling of functional trait values among species.

All composition and diversity metrics correspond to the median and 90\%
percentile obtained after 50 iterations of the taxonomy uncertainty
propagation framework and the gap filling process. The stochastic
trajectories were similarily obtained after 50 iterations of the random
sampling.

\section{Results}\label{results}

\subsection{Recruitment Diversity}\label{recruitment-diversity}

\subsubsection{Taxonomic Diversity}\label{taxonomic-diversity}

The diversity trajecctories of punctual recruitment followed a
consistent trajectory after disturbance with first an increase of the
richness and a decrease of the evenness (Figure (\ref{fig:DivTraj}). For
all disturbed plots,both richness and evenness tended to return towards
initial values but none had recovered 30 years after disturbance. The
accumulated recruits displayed sharp increasing richness (order 0) and
decreasing evenness (order 2) after intense disturbance (T3 and some T2,
Appendix I, fig. S1).

\begin{figure*}

{\centering \includegraphics[width=0.8\linewidth]{RecruitmentTrajectories_files/figure-latex/DivTraj-1} 

}

\caption{Trajectories of Richness, Shannon and Simpson diversity for 2-years laps punctual  recruitment (upper panels) and divergence to null model (lower panels). Lines colors refer to the perturbation regime: green for control, blue for T1, orange for T2 and red for T3 disturbance treatments. Plain lines correspond to the median observed after uncertainty propagation and are given along with the 95\% confidence interval (grey envelope).}\label{fig:DivTraj}
\end{figure*}

Punctual and accumulated recruitment diversities were then compared to
the stochastic trajectories of a random sampling. Richness (order 0) and
evenness (order 2) of punctual recruits remained equivalent or higher
than for a random sampling in control plots while both were lower in
disturbed plots. Disturbed plots however followed humped-shaped
trajectories heading towards a recovery of the initial state (Figure
\ref{fig:DivTraj}). Accumulated recruitment richness and evenness were
higher or equivalent to those of the random sampling after low
disturbance intensity (plots T1 and some plots T2) but lower after
intense disturbance (plots T3 and a plot T2, Appendix I fig. S1).

\subsubsection{Functional Diversity and
Composition}\label{functional-diversity-and-composition}

Communities functional diversity was measured with the Rao diversity and
compared to the stochastic trajectories of a random traits shuffling. In
disturbed plots (T2 and T3), the functional diversity decreased until 15
years after disturbance (Figure \ref{fig:FunTraj}) before recovering
towards the initial values. While the recovery was not achieved for the
most disturbed plots, the functional diversity of lighter disturbance
plots recovered faster and for some T1 plots exceeded the initial
values. For all plots, disturbed or not, the observed functional
diversity was lower than this of the random model, to the exception of
two plots T1.

\begin{figure}

{\centering \includegraphics{RecruitmentTrajectories_files/figure-latex/FunTraj-1} 

}

\caption{Functional diversity of punctual recruited trees from the considered functional traits and divergence to null model. Values reported correspond to the plot-level median and the 95\% confidence interval obtained after 50 repetition of the taxonomic uncertainty propagation and the functional database gap-filling processes and 50 run of the null model. Lines colors correspond to the logging treatment initially applied (green for control, blue for T1,orange for T2 and red for T3).}\label{fig:FunTraj}
\end{figure}

Trajectories of the functional traits showed a switch in disturbed plots
towards species with large exchange surface area, light tissues (high
SLA, low leaf toughness and thickness and low wood specific gravity)
with smaller maximum height (Figure \ref{fig:CWM}). Functional traits
either followed hump-shaped trajectories with an ongoing recovery or an
achieved return to the initial state (for SLA,Bark thickness and leaf
thickness and Hmax to a certain extent).

\begin{figure*}

{\centering \includegraphics[width=0.8\linewidth]{RecruitmentTrajectories_files/figure-latex/CWM-1} 

}

\caption{Community weighted means (CWM) of the four disturbance treatment for the four leaf traits, the two stem traits  and the specific Hmax. Values reported correspond to the plot-level median obtained after 50 repetition of the taxonomic uncertainty propagation and the functional database gap-filling processes. Lines colors correspond to the disturbance intensity (green for control, blue for T1,orange for T2 and red for T3).}\label{fig:CWM}
\end{figure*}

\subsection{Recruitment Turnover}\label{recruitment-turnover}

In control plots species turnover remained highly stable for the 30
sampled years (Figure \ref{fig:Turnover}), reflecting a strong
similarity between the initial plots composition and the punctual
recruits. In disturbed plots, the taxonomic turnover followed a marked
hump-backed trajectory, with a maximum value reached around 15 years
after disturbance and a maximum positively correlated to the disturbance
intensity (\(\rho_{spearman}=0.93\)). Thirty years after disturbance the
turnover of all disturbed plots had return to low values close to zero.

\begin{figure}

{\centering \includegraphics{RecruitmentTrajectories_files/figure-latex/Turnover-1} 

}

\caption{Trajectories over the 30 sampled years of the abundance-based turnover between recruited trees and intial communities before disturbance. Grey envelopes correspond to the 0.025 and 0.975 percentiles of the uncertainty propagation procedue and lines to the median in green for control, blue for T1,orange for T2 and red for T3).}\label{fig:Turnover}
\end{figure}

\section{Discussion}\label{discussion}

The composition, diversity, and functional trajectories in
post-disturbance times were analysed in the Paracou station to determine
the determinants of forests response to disturbance and their
resilience. The 30 years-long monitoring identified two distinct
recruitment phases, first defined by the stochastic recruitment of
already grown saplings and then driven by ecological rules restricting
the pool of recruited species on the basis of their functional strategy.
The second phase changed the functioning and diversity of communities
for more than 30 years after disturbance, disclaiming communities
functioning.

\subsection{On the underlyings of the hump-shaped
trajectories}\label{on-the-underlyings-of-the-hump-shaped-trajectories}

The trajectories of recruitment richness, key functional traits (SLA and
bark thickness) and the species turnover exhibited clear hump-shaped,
unimodal trajectories.

Trees recruited in the first place (0-8 years) resembled the
pre-disturbance communities and their functional diversity matched this
of a stochastic recruitment process. This first recruitment phase likely
involved already grown saplings (DBH \textless{}10cm) that germinated
before disturbance and immediatly benefitted from the increased
enlightment and the alleviated competition following diturbance
\citep{Herault2010}.

Then during a second phase the evenness and the functional diversity of
recruited trees decreased, translating a more restricted pool of
recruited species based on their functional strategy. Following intense
disturbance, sharp changes in the SLA, wood density and leaf thickness
trajectories then revealed the prominent recruitment of short-lived,
fast growing hard pionneers species with competitive and efficient light
acquisition (Figure \citet{ref}(fig:Turnover))
\citep{Wright2004, Chave2009b, Herault2011, Reich2014}.\\
The second phase therefore likely incorporated true recruits,
\emph{i.e.} trees germinated from the seed bank, submitted to
deterministic recruitment processes excluding the least competitive
species. Along time diversity trajectories returned towards the initial
state and matched the values of stochastic recruitment, reflecting a
decreasing importance of the deterministic recruitment processes. The
balance between deterministic and stochastic processes was determined by
the initial disturbance intensity. After light disturbance (T1 plots),
the recruitment matched the composition of the initial communities but
the pool of recruited species was restricted by the competitive
exclusion for resources. dominant species were more pioneers and light
demanders, with strategies of efficient resource acquisition (high SLA
and leaf chlorophyll content) and inexpensive, short-lived tissues (low
leaf thickness and thoughness, small Hmax and low wood density and bark
thickness). Although competitive exclusion decreased the functional
diversity of recruited trees, their eveness remained high, so the were
no high dominance of hard pionners
\citep{Hubbell1999, Sheil2003, Bongers2009}. This might due to
recruitment and dispersal limitation due to the short dispersal distance
observed for tropical trees, specifically in Paracou with the genetic
clumping of some pioneers \citep{Leclerc2015, Scotti2015a}. After
intense disturbance however, the recruitment composition rapidly
differed from the pre-disturbance state and corresponded to a sharp
increase for the SLA and bark thickness CWM, likely reflecting the
overhelming recruitment of hard pioneers. This recruitment lasted for 15
years after disturbance, that is the life expectancy of hard pionners,
and was followed by a progressive increase of the functional diversity.
The recruitment remained dominated by acquisitive functional strategies
but the initial composition and diversity progressively recovered.
Communities trajectories involved an interplay between stochastic and
deterministic recruitment, advocating that disturbance effectively
maintain forests diversity \citep{Molino2001, Sheil2003}.

\subsection{On the resilience of the recruitment
process}\label{on-the-resilience-of-the-recruitment-process}

After 30 years, although taxonomic and functional diversity had
recovered initial values, the recruitment rates remained higher and the
processes did not match the stochastic recruitment observed in
undisturbed plots. Recruitment processes might then be consistently
resilient, despite the settlment of long-lived pioneers after intense
disturbance, but slow to recover.

The recovery of recruitment processes involves the convergence of
communities towards the initial state. More than commonly thought, the
ecosystem trajectories in the taxonomic space would depend on the
pre-disturbance ecosystem characteristics and this concur to maintain
the initial taxonomic differences among local communities
\citep{Anderson200}. In contrast the trajectories of traits and
functional diversity were essentially similar among treatments, arguing
for the confluence of communities in the functional space despite their
divergence in taxonomic composition \citep{Fukami2005}. This confirms
previous results from the Paracou experiment, conducted 10 years
\citep{Molino2001} and 20 years \citep{Baraloto2012a} after disturbance,
where the early signs of the resilience of taxonomic and functional
composition had been detected. The time length, i.e.~several decades, of
the recovery processes entails great caution regarding the forest
conservation and exploitation guidelines if the pursued objectives are a
complete recovery of pre-disturbance ecosystem properties.

????To that was added the involvement of the seed bank in the recovery
processes which own resilience remains unknown.

????Any storage effect, altering the pool of recruitable species, would
modify the resilience of the community itself and impact the response to
further disturbance events \citep{Norden2009}.

????In such case, the competitive exclusion among dormant life-stage
(seeds or even seedlings) would be harsher and likely bring more radical
changes in the recruitment composition and functional profile of the
community.

\section{Conclusion}\label{conclusion}

The 30 years monitoring of the Paracou plots highlighted the tropical
forests' response to disturbance composed of two recruitment phase
modulated by the disturbance intensity. In the short-term forests
response was driven by the enhanced growth of grown saplings benefiting
from the alleviated competition and the environmental changes. Above an
intensity threshold the recruitment was besides dominated by
hard-pioneers radically changing the recruitment composition, diversity
and, likely, functioning. In the long-term response was driven by
recruits from the seed bank which underwent selection towards light
demanding species and similarity limitation enhancing the functional
diversity. These deterministic processes followed a gradual balance with
the stochastic recruitment of mature forests which eventually restored
communities diversity and composition, maintaining their initial
differences. Although forests proved resilient to intense disturbance
this appeared to be a long-term processes likely only valid for single
disturbance events.

\begin{center}\rule{0.5\linewidth}{\linethickness}\end{center}

%----------------------------------------------------------------------------------------
%	REFERENCE LIST
%----------------------------------------------------------------------------------------

\bibliographystyle{mee}
\makeatletter
% The filename has .bib extension the must be eliminated
\filename@parse{references.bib}
% parse stores the file name in base. Extension starts at the first dot, so don't use dots in file names.
\bibliography{\filename@base}
\makeatother


%----------------------------------------------------------------------------------------

\end{document}
