%%%%%%%%%%%%%%%%%%%%%%%%%%%%%%%%%%%%%%%%%
% Article EcoFoG
% Version 2.1 (23/10/2017)
%
% adapté de :
% Stylish Article
% LaTeX Template
% Version 1.0 (31/1/13)
%
% This template has been downloaded from:
% http://www.LaTeXTemplates.com
%
% Original author:
% Mathias Legrand (legrand.mathias@gmail.com)
%
% License:
% CC BY-NC-SA 3.0 (http://creativecommons.org/licenses/by-nc-sa/3.0/)
%
%%%%%%%%%%%%%%%%%%%%%%%%%%%%%%%%%%%%%%%%%


%----------------------------------------------------------------------------------------
%	PACKAGES AND OTHER DOCUMENT CONFIGURATIONS
%----------------------------------------------------------------------------------------

\documentclass[fleqn,10pt]{ArtEcoFoG} % Document font size and equations flushed left

\setcounter{tocdepth}{3} % Show only three levels in the table of contents section: sections, subsections and subsubsections


% Pandoc environments
\usepackage{framed}
\usepackage{fancyvrb}
\providecommand{\tightlist}{%
  \setlength{\itemsep}{0pt}\setlength{\parskip}{0pt}}
\newcommand{\VerbBar}{|}
\newcommand{\VERB}{\Verb[commandchars=\\\{\}]}
\DefineVerbatimEnvironment{Highlighting}{Verbatim}{commandchars=\\\{\}, fontsize=\scriptsize} % Code R
\definecolor{shadecolor}{RGB}{248,248,248}
\newenvironment{Shaded}{\begin{snugshade}}{\end{snugshade}}
\newcommand{\KeywordTok}[1]{\textcolor[rgb]{0.13,0.29,0.53}{\textbf{{#1}}}}
\newcommand{\DataTypeTok}[1]{\textcolor[rgb]{0.13,0.29,0.53}{{#1}}}
\newcommand{\DecValTok}[1]{\textcolor[rgb]{0.00,0.00,0.81}{{#1}}}
\newcommand{\BaseNTok}[1]{\textcolor[rgb]{0.00,0.00,0.81}{{#1}}}
\newcommand{\FloatTok}[1]{\textcolor[rgb]{0.00,0.00,0.81}{{#1}}}
\newcommand{\ConstantTok}[1]{\textcolor[rgb]{0.00,0.00,0.00}{{#1}}}
\newcommand{\CharTok}[1]{\textcolor[rgb]{0.31,0.60,0.02}{{#1}}}
\newcommand{\SpecialCharTok}[1]{\textcolor[rgb]{0.00,0.00,0.00}{{#1}}}
\newcommand{\StringTok}[1]{\textcolor[rgb]{0.31,0.60,0.02}{{#1}}}
\newcommand{\VerbatimStringTok}[1]{\textcolor[rgb]{0.31,0.60,0.02}{{#1}}}
\newcommand{\SpecialStringTok}[1]{\textcolor[rgb]{0.31,0.60,0.02}{{#1}}}
\newcommand{\ImportTok}[1]{{#1}}
\newcommand{\CommentTok}[1]{\textcolor[rgb]{0.56,0.35,0.01}{\textit{{#1}}}}
\newcommand{\DocumentationTok}[1]{\textcolor[rgb]{0.56,0.35,0.01}{\textbf{\textit{{#1}}}}}
\newcommand{\AnnotationTok}[1]{\textcolor[rgb]{0.56,0.35,0.01}{\textbf{\textit{{#1}}}}}
\newcommand{\CommentVarTok}[1]{\textcolor[rgb]{0.56,0.35,0.01}{\textbf{\textit{{#1}}}}}
\newcommand{\OtherTok}[1]{\textcolor[rgb]{0.56,0.35,0.01}{{#1}}}
\newcommand{\FunctionTok}[1]{\textcolor[rgb]{0.00,0.00,0.00}{{#1}}}
\newcommand{\VariableTok}[1]{\textcolor[rgb]{0.00,0.00,0.00}{{#1}}}
\newcommand{\ControlFlowTok}[1]{\textcolor[rgb]{0.13,0.29,0.53}{\textbf{{#1}}}}
\newcommand{\OperatorTok}[1]{\textcolor[rgb]{0.81,0.36,0.00}{\textbf{{#1}}}}
\newcommand{\BuiltInTok}[1]{{#1}}
\newcommand{\ExtensionTok}[1]{{#1}}
\newcommand{\PreprocessorTok}[1]{\textcolor[rgb]{0.56,0.35,0.01}{\textit{{#1}}}}
\newcommand{\AttributeTok}[1]{\textcolor[rgb]{0.77,0.63,0.00}{{#1}}}
\newcommand{\RegionMarkerTok}[1]{{#1}}
\newcommand{\InformationTok}[1]{\textcolor[rgb]{0.56,0.35,0.01}{\textbf{\textit{{#1}}}}}
\newcommand{\WarningTok}[1]{\textcolor[rgb]{0.56,0.35,0.01}{\textbf{\textit{{#1}}}}}
\newcommand{\AlertTok}[1]{\textcolor[rgb]{0.94,0.16,0.16}{{#1}}}
\newcommand{\ErrorTok}[1]{\textcolor[rgb]{0.64,0.00,0.00}{\textbf{{#1}}}}
\newcommand{\NormalTok}[1]{{#1}}
\usepackage{longtable,booktabs}
\usepackage{caption}
% These lines are needed to make table captions work with longtable:
\makeatletter
\def\fnum@table{\tablename~\thetable}
\makeatother
% longtable 2 columns
% https://tex.stackexchange.com/questions/161431/how-to-solve-longtable-is-not-in-1-column-mode-error
\makeatletter
\let\oldlt\longtable
\let\endoldlt\endlongtable
\def\longtable{\@ifnextchar[\longtable@i \longtable@ii}
\def\longtable@i[#1]{\begin{figure}[t]
\onecolumn
\begin{minipage}{0.5\textwidth}\scriptsize
\oldlt[#1]
}
\def\longtable@ii{\begin{figure}[t]
\onecolumn
\begin{minipage}{0.5\textwidth}\scriptsize
\oldlt
}
\def\endlongtable{\endoldlt
\end{minipage}
\twocolumn
\end{figure}}
\makeatother

\usepackage{graphicx,grffile}
\makeatletter
\def\maxwidth{\ifdim\Gin@nat@width>\linewidth\linewidth\else\Gin@nat@width\fi}
\def\maxheight{\ifdim\Gin@nat@height>\textheight0.8\textheight\else\Gin@nat@height\fi}
\makeatother
% Scale images if necessary, so that they will not overflow the page
% margins by default, and it is still possible to overwrite the defaults
% using explicit options in \includegraphics[width, height, ...]{}
\setkeys{Gin}{width=\maxwidth,height=\maxheight,keepaspectratio}

% User-adder preamble
\usepackage{amsmath}

%----------------------------------------------------------------------------------------
%	ARTICLE INFORMATION
%----------------------------------------------------------------------------------------

\JournalInfo{Hal 00679993} % Journal information
\Archive{DOI xxxx} % Additional notes (e.g. copyright, DOI, review/research article)

\PaperTitle{Titre de l'article} % Article title

\Authors{
Prénom Nom\textsuperscript{1*}\\ Deuxième Auteur\textsuperscript{2}
} % Authors
\affiliation{
\textsuperscript{1}UMR EcoFoG, AgroParistech, CNRS, Cirad, INRA, Université des Antilles,
Université de Guyane.\\ \hspace{1em} Campus Agronomique, 97310 Kourou, France.\\\textsuperscript{2}Department of Ecology, University of Edimburgh\\ \hspace{1em} Street address, Zip code, Country.
}
\affiliation{*\textbf{Contact}: prenom.nom@ecofog.gf, http://www.ecofog.gf/spip.php?article47} % Corresponding author

\Keywords{mot-clés, séparés par des virgules} % Keywords - if you don't want any simply remove all the text between the curly brackets
\newcommand{\keywordname}{Mots-clés} % Defines the keywords heading name

%----------------------------------------------------------------------------------------
%	ABSTRACT
%----------------------------------------------------------------------------------------

\Abstract{
Résumé de l'article.
}

%----------------------------------------------------------------------------------------

\begin{document}

\selectlanguage{french}

\flushbottom % Makes all text pages the same height

\maketitle % Print the title and abstract box

\tableofcontents % Print the contents section

\thispagestyle{empty} % Removes page numbering from the first page

%----------------------------------------------------------------------------------------
%	ARTICLE CONTENTS
%----------------------------------------------------------------------------------------


\section{Introduction}\label{introduction}

The main determinant identified and retained over decades of studies on
forest dynamics rely on the demographic processes determining trees
recruitment. As stated by the Intermediate Disturbance Hypothesis (IDH),
a high biodiversity in communities would be maintained by the
variability of environmental conditions which would favor and enhance
the growth of a variety of species along time which makes the recruited
trees to be very diverse. Similarly, the high diversity of tropical
forests is assumed to be maintained by the tree fall gaps dynamics which
impulses a different recruitment regime between enlightened and dyer
gaps and shaded, wet undercover areas. Recruitment dynamics are thus
crucial to understand, as determinant of the forest-to-be and driver of
community assemblage. A large literature investigated forests population
dynamics and successfully modeled forest response to disturbance. Based
on stands structure observed in the field, these models proved relevant
predictors of carbon sequestration, commercial stocks and diameter
distribution dynamics in the short and in the long term
\ref{@Gourlet-Fleury2005}. Predicting in this way species population
recovery and subsequently stands diversity after disturbance however
seems tricky as the growth and survival dynamics in the early life
stage, far from being well understood, are highly variable among species
which makes the recruitment of a tree too infrequent an event and really
poorly correlated to the presence of similar species at the time
considered. Attempts were successfully conducted to model the
regeneration of single species however the response of the whole
recruited community after disturbance could not reasonably be assessed
via similar methods which would require the combination of accurate
knowledge of each species ecology and behavior.

\begin{itemize}
\item
  Post logging dynamics have been identified but the underlying
  mechanisms remain unclear, and it is interesting because it gives
  clues on the functioning of forests (which filters are at stake and
  how they apply on the ecosystems)
\item
  We were interesting in the successional rate along which recruitment
  functional diversity varies: do we already see the new generation of
  functional type selected, compared to mature (old growth) forest
\item
  Do the stand return to initial condition, is the cycle achieved?
\item
  Hypotheses:
\end{itemize}

\begin{enumerate}
\def\labelenumi{\arabic{enumi})}
\item
  The more intense the logging, the most significant the impact on
  composition and diversity metrics.
\item
  We expect an increase in the stand diversity (Intermediate disturbance
  hypothesis) or no change in richness, at least for low logging
  intensity
\item
  Exploitation would change the functional composition toward more
  light-demanding strategies
\item
  We expect the diversity to follow a cyclic dynamic, as observed for
  stands biomass, and the ecosystem functioning to be resilient, at
  least for low intensity treatment
\end{enumerate}

\section{Material and Methods}\label{material-and-methods}

\emph{Study Site} Analyses were based on the inventories conducted at
the Paracou station in French Guiana (5°18'N and 52°53'W) in a lowland
tropical rain forest. The site corresponds to a tropical wet climate
with mean annual precipitation averaging 2980 mm.y-1 (30-y period) with
a 3-months dry season (\textless{} 100 mm.months-1) from mid-August to
mid-November, and a one-month dry season in march. Elevation ranges
between 5 and 50 m and mean annual temperature is 26°C. Soils correspond
to thin acrisols over a layer of transformed saprolite with low
permeability generating lateral drainage during heavy rains. The
experiment corresponds to a network of twelve 6.25ha plots where three
logging treatments representing a gradient of logging intensity were
applied from 1985 to 1987. Logging treatments were attributed according
to a randomized plot design with three replicate blocks of four plots.
The logging treatments correspond to averages of 10 trees remove per
hectare with a diameter at 1.3 m height (DBH) above 50 cm for treatment
1 (T1), 32 trees/ha above 40 cm DBH for treatment 2 (T2) and 40 trees
above 40 cm DBH for treatment 3 (T3). Treatments T2 and T3 included
besides logging of 10trees/ha the thinning by poison girdling.

\emph{Inventories Protocol and Dataset Collection} The study site
corresponds to a tropical rain forest with a dominance of Fabaceae,
Chrysobalanaceae, Lecythidaceae and Sapotaceae. In the twelve
experimental plots all trees above 10 cm DBH are mapped and measured
annually since 1984, which allows to survey the trees recruited
annually. During the inventories trees are first identified with a
vernacular name, assigned by the field team, and then with a scientific
name, assigned by a botanist during regular botanical campaigns. In
1984, specific vernacular names were given to 62 commercial or common
species whereas two identifiers separating palm trees and trees were
given to all other species. Botanical campaigns to identify all trees at
the species level started in 2003 and continued every 5 to 6. In total
from 1984 to 2016 around 850 species were identified in the 12 plots.
For the trees that were identified at the species level vernacular names
attributed in the first place generate some taxonomic uncertainty as
they may correspond to a variety of botanical species. To account for
this taxonomic uncertainty in the composition and diversity metrics we
used a Bayesian framework estimating the expectancy and variance of
composition and diversity metrics. The frameworks is based on a set of
complete inventories extrapolated from the initial one through the
replacement of vernacular name by botanical species. These replacement
were made through a multinomial distribution
M\textsubscript{v}({[}s\textsubscript{1}, s\textsubscript{2}, \ldots{},
s\textsubscript{N}{]} ,{[}α\textsubscript{1},
α\textsubscript{2},\ldots{}, α\textsubscript{N}{]}) based on the
observed association probability {[}α\textsubscript{1},
α\textsubscript{2},\ldots{}, α\textsubscript{N}{]} between each
vernacular name v and the species recorded in the inventory {[}s1, s2,
\ldots{}, sN{]}. See appendix 1 for the detailed methodology. Diversity
metrics correspond to the expectancy obtained after 50 iteration of the
taxonomic uncertainty propagation framework.

\emph{Recruitment Survey} To assess the demographic dynamics along the
30 sampled year we divided the inventoried trees between trees recorded
before the logging in 1984 and the trees recruited afterward. Analysis
first surveyed the communities of trees recruited every 5 years-laps
after exploitation, hereafter called ``punctual recruitment''. Then to
assess the diversity of the stands under settlement analysis surveyed
the communities of recruited trees accumulated since exploitation,
hereafter called ``accumulated recruits''. To tackle the unequal number
of recruited trees between treatments, which would mislead the diversity
with the effect of sampling size, the diversity was assessed as the
average measure obtained from the resampling of 50 trees in the 5 years
-- laps recruited communities. The diversity of accumulated recruits
corresponded to the accumulation of resampled trees. Taxonomic diversity
was assessed through Richness and the Hill number translation of Shannon
and Simpson indices. These three indices belong to the set of HCDT or
generalized entropy, respectively corresponding to the 0, 1 and 2 order
of diversity (q). This order q grasps the balance between richness and
evenness in the community as it determines the emphasis on common
species in the diversity metric, with common species weighting more than
rare ones when q increases. The similarity between the recruited trees
and the old growth forest was measured with the turnover metrics
detailed in \citet{Podani2013a}. The metric used correspond to the
relativized abundance replacement, the sum of abundance in one site that
is replaced by completely different species, normalized by the maximum
abundance shared by the two communities.
T\_ab=(∑\_(i=1)\^{}n▒\textbar{}x\_ia 〖-x〗\emph{ib \textbar{}
-\textbar{}∑}(i=1)\textsuperscript{n▒〖x\_ia-∑\emph{(i=1)\^{}n▒x\_ib
〗\textbar{})/(∑}(i=1)}n▒max\{x\_ia;x\_ib \} )

To determine whether trees recruitment ensued from a random process the
observed diversity trajectories were compared to those generated by 50
repetitions of stochastic null models. The null model for taxonomic
diversity randomly shuffled individuals among plots while preserving
species abundance and plots' tree density.

%----------------------------------------------------------------------------------------
%	REFERENCE LIST
%----------------------------------------------------------------------------------------

\bibliographystyle{mee}
\bibliography{references}

%----------------------------------------------------------------------------------------

\end{document}
